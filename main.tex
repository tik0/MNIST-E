\documentclass{article} % For LaTeX2e
\usepackage{iclr2019_conference,times}

% Optional math commands from https://github.com/goodfeli/dlbook_notation.
%%%%% NEW MATH DEFINITIONS %%%%%

\usepackage{amsmath,amsfonts,bm}

% Mark sections of captions for referring to divisions of figures
\newcommand{\figleft}{{\em (Left)}}
\newcommand{\figcenter}{{\em (Center)}}
\newcommand{\figright}{{\em (Right)}}
\newcommand{\figtop}{{\em (Top)}}
\newcommand{\figbottom}{{\em (Bottom)}}
\newcommand{\captiona}{{\em (a)}}
\newcommand{\captionb}{{\em (b)}}
\newcommand{\captionc}{{\em (c)}}
\newcommand{\captiond}{{\em (d)}}

% Highlight a newly defined term
\newcommand{\newterm}[1]{{\bf #1}}


% Figure reference, lower-case.
\def\figref#1{figure~\ref{#1}}
% Figure reference, capital. For start of sentence
\def\Figref#1{Figure~\ref{#1}}
\def\twofigref#1#2{figures \ref{#1} and \ref{#2}}
\def\quadfigref#1#2#3#4{figures \ref{#1}, \ref{#2}, \ref{#3} and \ref{#4}}
% Section reference, lower-case.
\def\secref#1{section~\ref{#1}}
% Section reference, capital.
\def\Secref#1{Section~\ref{#1}}
% Reference to two sections.
\def\twosecrefs#1#2{sections \ref{#1} and \ref{#2}}
% Reference to three sections.
\def\secrefs#1#2#3{sections \ref{#1}, \ref{#2} and \ref{#3}}
% Reference to an equation, lower-case.
\def\eqref#1{equation~\ref{#1}}
% Reference to an equation, upper case
\def\Eqref#1{Equation~\ref{#1}}
% A raw reference to an equation---avoid using if possible
\def\plaineqref#1{\ref{#1}}
% Reference to a chapter, lower-case.
\def\chapref#1{chapter~\ref{#1}}
% Reference to an equation, upper case.
\def\Chapref#1{Chapter~\ref{#1}}
% Reference to a range of chapters
\def\rangechapref#1#2{chapters\ref{#1}--\ref{#2}}
% Reference to an algorithm, lower-case.
\def\algref#1{algorithm~\ref{#1}}
% Reference to an algorithm, upper case.
\def\Algref#1{Algorithm~\ref{#1}}
\def\twoalgref#1#2{algorithms \ref{#1} and \ref{#2}}
\def\Twoalgref#1#2{Algorithms \ref{#1} and \ref{#2}}
% Reference to a part, lower case
\def\partref#1{part~\ref{#1}}
% Reference to a part, upper case
\def\Partref#1{Part~\ref{#1}}
\def\twopartref#1#2{parts \ref{#1} and \ref{#2}}

\def\ceil#1{\lceil #1 \rceil}
\def\floor#1{\lfloor #1 \rfloor}
\def\1{\bm{1}}
\newcommand{\train}{\mathcal{D}}
\newcommand{\valid}{\mathcal{D_{\mathrm{valid}}}}
\newcommand{\test}{\mathcal{D_{\mathrm{test}}}}

\def\eps{{\epsilon}}


% Random variables
\def\reta{{\textnormal{$\eta$}}}
\def\ra{{\textnormal{a}}}
\def\rb{{\textnormal{b}}}
\def\rc{{\textnormal{c}}}
\def\rd{{\textnormal{d}}}
\def\re{{\textnormal{e}}}
\def\rf{{\textnormal{f}}}
\def\rg{{\textnormal{g}}}
\def\rh{{\textnormal{h}}}
\def\ri{{\textnormal{i}}}
\def\rj{{\textnormal{j}}}
\def\rk{{\textnormal{k}}}
\def\rl{{\textnormal{l}}}
% rm is already a command, just don't name any random variables m
\def\rn{{\textnormal{n}}}
\def\ro{{\textnormal{o}}}
\def\rp{{\textnormal{p}}}
\def\rq{{\textnormal{q}}}
\def\rr{{\textnormal{r}}}
\def\rs{{\textnormal{s}}}
\def\rt{{\textnormal{t}}}
\def\ru{{\textnormal{u}}}
\def\rv{{\textnormal{v}}}
\def\rw{{\textnormal{w}}}
\def\rx{{\textnormal{x}}}
\def\ry{{\textnormal{y}}}
\def\rz{{\textnormal{z}}}

% Random vectors
\def\rvepsilon{{\mathbf{\epsilon}}}
\def\rvtheta{{\mathbf{\theta}}}
\def\rva{{\mathbf{a}}}
\def\rvb{{\mathbf{b}}}
\def\rvc{{\mathbf{c}}}
\def\rvd{{\mathbf{d}}}
\def\rve{{\mathbf{e}}}
\def\rvf{{\mathbf{f}}}
\def\rvg{{\mathbf{g}}}
\def\rvh{{\mathbf{h}}}
\def\rvu{{\mathbf{i}}}
\def\rvj{{\mathbf{j}}}
\def\rvk{{\mathbf{k}}}
\def\rvl{{\mathbf{l}}}
\def\rvm{{\mathbf{m}}}
\def\rvn{{\mathbf{n}}}
\def\rvo{{\mathbf{o}}}
\def\rvp{{\mathbf{p}}}
\def\rvq{{\mathbf{q}}}
\def\rvr{{\mathbf{r}}}
\def\rvs{{\mathbf{s}}}
\def\rvt{{\mathbf{t}}}
\def\rvu{{\mathbf{u}}}
\def\rvv{{\mathbf{v}}}
\def\rvw{{\mathbf{w}}}
\def\rvx{{\mathbf{x}}}
\def\rvy{{\mathbf{y}}}
\def\rvz{{\mathbf{z}}}

% Elements of random vectors
\def\erva{{\textnormal{a}}}
\def\ervb{{\textnormal{b}}}
\def\ervc{{\textnormal{c}}}
\def\ervd{{\textnormal{d}}}
\def\erve{{\textnormal{e}}}
\def\ervf{{\textnormal{f}}}
\def\ervg{{\textnormal{g}}}
\def\ervh{{\textnormal{h}}}
\def\ervi{{\textnormal{i}}}
\def\ervj{{\textnormal{j}}}
\def\ervk{{\textnormal{k}}}
\def\ervl{{\textnormal{l}}}
\def\ervm{{\textnormal{m}}}
\def\ervn{{\textnormal{n}}}
\def\ervo{{\textnormal{o}}}
\def\ervp{{\textnormal{p}}}
\def\ervq{{\textnormal{q}}}
\def\ervr{{\textnormal{r}}}
\def\ervs{{\textnormal{s}}}
\def\ervt{{\textnormal{t}}}
\def\ervu{{\textnormal{u}}}
\def\ervv{{\textnormal{v}}}
\def\ervw{{\textnormal{w}}}
\def\ervx{{\textnormal{x}}}
\def\ervy{{\textnormal{y}}}
\def\ervz{{\textnormal{z}}}

% Random matrices
\def\rmA{{\mathbf{A}}}
\def\rmB{{\mathbf{B}}}
\def\rmC{{\mathbf{C}}}
\def\rmD{{\mathbf{D}}}
\def\rmE{{\mathbf{E}}}
\def\rmF{{\mathbf{F}}}
\def\rmG{{\mathbf{G}}}
\def\rmH{{\mathbf{H}}}
\def\rmI{{\mathbf{I}}}
\def\rmJ{{\mathbf{J}}}
\def\rmK{{\mathbf{K}}}
\def\rmL{{\mathbf{L}}}
\def\rmM{{\mathbf{M}}}
\def\rmN{{\mathbf{N}}}
\def\rmO{{\mathbf{O}}}
\def\rmP{{\mathbf{P}}}
\def\rmQ{{\mathbf{Q}}}
\def\rmR{{\mathbf{R}}}
\def\rmS{{\mathbf{S}}}
\def\rmT{{\mathbf{T}}}
\def\rmU{{\mathbf{U}}}
\def\rmV{{\mathbf{V}}}
\def\rmW{{\mathbf{W}}}
\def\rmX{{\mathbf{X}}}
\def\rmY{{\mathbf{Y}}}
\def\rmZ{{\mathbf{Z}}}

% Elements of random matrices
\def\ermA{{\textnormal{A}}}
\def\ermB{{\textnormal{B}}}
\def\ermC{{\textnormal{C}}}
\def\ermD{{\textnormal{D}}}
\def\ermE{{\textnormal{E}}}
\def\ermF{{\textnormal{F}}}
\def\ermG{{\textnormal{G}}}
\def\ermH{{\textnormal{H}}}
\def\ermI{{\textnormal{I}}}
\def\ermJ{{\textnormal{J}}}
\def\ermK{{\textnormal{K}}}
\def\ermL{{\textnormal{L}}}
\def\ermM{{\textnormal{M}}}
\def\ermN{{\textnormal{N}}}
\def\ermO{{\textnormal{O}}}
\def\ermP{{\textnormal{P}}}
\def\ermQ{{\textnormal{Q}}}
\def\ermR{{\textnormal{R}}}
\def\ermS{{\textnormal{S}}}
\def\ermT{{\textnormal{T}}}
\def\ermU{{\textnormal{U}}}
\def\ermV{{\textnormal{V}}}
\def\ermW{{\textnormal{W}}}
\def\ermX{{\textnormal{X}}}
\def\ermY{{\textnormal{Y}}}
\def\ermZ{{\textnormal{Z}}}

% Vectors
\def\vzero{{\bm{0}}}
\def\vone{{\bm{1}}}
\def\vmu{{\bm{\mu}}}
\def\vtheta{{\bm{\theta}}}
\def\va{{\bm{a}}}
\def\vb{{\bm{b}}}
\def\vc{{\bm{c}}}
\def\vd{{\bm{d}}}
\def\ve{{\bm{e}}}
\def\vf{{\bm{f}}}
\def\vg{{\bm{g}}}
\def\vh{{\bm{h}}}
\def\vi{{\bm{i}}}
\def\vj{{\bm{j}}}
\def\vk{{\bm{k}}}
\def\vl{{\bm{l}}}
\def\vm{{\bm{m}}}
\def\vn{{\bm{n}}}
\def\vo{{\bm{o}}}
\def\vp{{\bm{p}}}
\def\vq{{\bm{q}}}
\def\vr{{\bm{r}}}
\def\vs{{\bm{s}}}
\def\vt{{\bm{t}}}
\def\vu{{\bm{u}}}
\def\vv{{\bm{v}}}
\def\vw{{\bm{w}}}
\def\vx{{\bm{x}}}
\def\vy{{\bm{y}}}
\def\vz{{\bm{z}}}

% Elements of vectors
\def\evalpha{{\alpha}}
\def\evbeta{{\beta}}
\def\evepsilon{{\epsilon}}
\def\evlambda{{\lambda}}
\def\evomega{{\omega}}
\def\evmu{{\mu}}
\def\evpsi{{\psi}}
\def\evsigma{{\sigma}}
\def\evtheta{{\theta}}
\def\eva{{a}}
\def\evb{{b}}
\def\evc{{c}}
\def\evd{{d}}
\def\eve{{e}}
\def\evf{{f}}
\def\evg{{g}}
\def\evh{{h}}
\def\evi{{i}}
\def\evj{{j}}
\def\evk{{k}}
\def\evl{{l}}
\def\evm{{m}}
\def\evn{{n}}
\def\evo{{o}}
\def\evp{{p}}
\def\evq{{q}}
\def\evr{{r}}
\def\evs{{s}}
\def\evt{{t}}
\def\evu{{u}}
\def\evv{{v}}
\def\evw{{w}}
\def\evx{{x}}
\def\evy{{y}}
\def\evz{{z}}

% Matrix
\def\mA{{\bm{A}}}
\def\mB{{\bm{B}}}
\def\mC{{\bm{C}}}
\def\mD{{\bm{D}}}
\def\mE{{\bm{E}}}
\def\mF{{\bm{F}}}
\def\mG{{\bm{G}}}
\def\mH{{\bm{H}}}
\def\mI{{\bm{I}}}
\def\mJ{{\bm{J}}}
\def\mK{{\bm{K}}}
\def\mL{{\bm{L}}}
\def\mM{{\bm{M}}}
\def\mN{{\bm{N}}}
\def\mO{{\bm{O}}}
\def\mP{{\bm{P}}}
\def\mQ{{\bm{Q}}}
\def\mR{{\bm{R}}}
\def\mS{{\bm{S}}}
\def\mT{{\bm{T}}}
\def\mU{{\bm{U}}}
\def\mV{{\bm{V}}}
\def\mW{{\bm{W}}}
\def\mX{{\bm{X}}}
\def\mY{{\bm{Y}}}
\def\mZ{{\bm{Z}}}
\def\mBeta{{\bm{\beta}}}
\def\mPhi{{\bm{\Phi}}}
\def\mLambda{{\bm{\Lambda}}}
\def\mSigma{{\bm{\Sigma}}}

% Tensor
\DeclareMathAlphabet{\mathsfit}{\encodingdefault}{\sfdefault}{m}{sl}
\SetMathAlphabet{\mathsfit}{bold}{\encodingdefault}{\sfdefault}{bx}{n}
\newcommand{\tens}[1]{\bm{\mathsfit{#1}}}
\def\tA{{\tens{A}}}
\def\tB{{\tens{B}}}
\def\tC{{\tens{C}}}
\def\tD{{\tens{D}}}
\def\tE{{\tens{E}}}
\def\tF{{\tens{F}}}
\def\tG{{\tens{G}}}
\def\tH{{\tens{H}}}
\def\tI{{\tens{I}}}
\def\tJ{{\tens{J}}}
\def\tK{{\tens{K}}}
\def\tL{{\tens{L}}}
\def\tM{{\tens{M}}}
\def\tN{{\tens{N}}}
\def\tO{{\tens{O}}}
\def\tP{{\tens{P}}}
\def\tQ{{\tens{Q}}}
\def\tR{{\tens{R}}}
\def\tS{{\tens{S}}}
\def\tT{{\tens{T}}}
\def\tU{{\tens{U}}}
\def\tV{{\tens{V}}}
\def\tW{{\tens{W}}}
\def\tX{{\tens{X}}}
\def\tY{{\tens{Y}}}
\def\tZ{{\tens{Z}}}


% Graph
\def\gA{{\mathcal{A}}}
\def\gB{{\mathcal{B}}}
\def\gC{{\mathcal{C}}}
\def\gD{{\mathcal{D}}}
\def\gE{{\mathcal{E}}}
\def\gF{{\mathcal{F}}}
\def\gG{{\mathcal{G}}}
\def\gH{{\mathcal{H}}}
\def\gI{{\mathcal{I}}}
\def\gJ{{\mathcal{J}}}
\def\gK{{\mathcal{K}}}
\def\gL{{\mathcal{L}}}
\def\gM{{\mathcal{M}}}
\def\gN{{\mathcal{N}}}
\def\gO{{\mathcal{O}}}
\def\gP{{\mathcal{P}}}
\def\gQ{{\mathcal{Q}}}
\def\gR{{\mathcal{R}}}
\def\gS{{\mathcal{S}}}
\def\gT{{\mathcal{T}}}
\def\gU{{\mathcal{U}}}
\def\gV{{\mathcal{V}}}
\def\gW{{\mathcal{W}}}
\def\gX{{\mathcal{X}}}
\def\gY{{\mathcal{Y}}}
\def\gZ{{\mathcal{Z}}}

% Sets
\def\sA{{\mathbb{A}}}
\def\sB{{\mathbb{B}}}
\def\sC{{\mathbb{C}}}
\def\sD{{\mathbb{D}}}
% Don't use a set called E, because this would be the same as our symbol
% for expectation.
\def\sF{{\mathbb{F}}}
\def\sG{{\mathbb{G}}}
\def\sH{{\mathbb{H}}}
\def\sI{{\mathbb{I}}}
\def\sJ{{\mathbb{J}}}
\def\sK{{\mathbb{K}}}
\def\sL{{\mathbb{L}}}
\def\sM{{\mathbb{M}}}
\def\sN{{\mathbb{N}}}
\def\sO{{\mathbb{O}}}
\def\sP{{\mathbb{P}}}
\def\sQ{{\mathbb{Q}}}
\def\sR{{\mathbb{R}}}
\def\sS{{\mathbb{S}}}
\def\sT{{\mathbb{T}}}
\def\sU{{\mathbb{U}}}
\def\sV{{\mathbb{V}}}
\def\sW{{\mathbb{W}}}
\def\sX{{\mathbb{X}}}
\def\sY{{\mathbb{Y}}}
\def\sZ{{\mathbb{Z}}}

% Entries of a matrix
\def\emLambda{{\Lambda}}
\def\emA{{A}}
\def\emB{{B}}
\def\emC{{C}}
\def\emD{{D}}
\def\emE{{E}}
\def\emF{{F}}
\def\emG{{G}}
\def\emH{{H}}
\def\emI{{I}}
\def\emJ{{J}}
\def\emK{{K}}
\def\emL{{L}}
\def\emM{{M}}
\def\emN{{N}}
\def\emO{{O}}
\def\emP{{P}}
\def\emQ{{Q}}
\def\emR{{R}}
\def\emS{{S}}
\def\emT{{T}}
\def\emU{{U}}
\def\emV{{V}}
\def\emW{{W}}
\def\emX{{X}}
\def\emY{{Y}}
\def\emZ{{Z}}
\def\emSigma{{\Sigma}}

% entries of a tensor
% Same font as tensor, without \bm wrapper
\newcommand{\etens}[1]{\mathsfit{#1}}
\def\etLambda{{\etens{\Lambda}}}
\def\etA{{\etens{A}}}
\def\etB{{\etens{B}}}
\def\etC{{\etens{C}}}
\def\etD{{\etens{D}}}
\def\etE{{\etens{E}}}
\def\etF{{\etens{F}}}
\def\etG{{\etens{G}}}
\def\etH{{\etens{H}}}
\def\etI{{\etens{I}}}
\def\etJ{{\etens{J}}}
\def\etK{{\etens{K}}}
\def\etL{{\etens{L}}}
\def\etM{{\etens{M}}}
\def\etN{{\etens{N}}}
\def\etO{{\etens{O}}}
\def\etP{{\etens{P}}}
\def\etQ{{\etens{Q}}}
\def\etR{{\etens{R}}}
\def\etS{{\etens{S}}}
\def\etT{{\etens{T}}}
\def\etU{{\etens{U}}}
\def\etV{{\etens{V}}}
\def\etW{{\etens{W}}}
\def\etX{{\etens{X}}}
\def\etY{{\etens{Y}}}
\def\etZ{{\etens{Z}}}

% The true underlying data generating distribution
\newcommand{\pdata}{p_{\rm{data}}}
% The empirical distribution defined by the training set
\newcommand{\ptrain}{\hat{p}_{\rm{data}}}
\newcommand{\Ptrain}{\hat{P}_{\rm{data}}}
% The model distribution
\newcommand{\pmodel}{p_{\rm{model}}}
\newcommand{\Pmodel}{P_{\rm{model}}}
\newcommand{\ptildemodel}{\tilde{p}_{\rm{model}}}
% Stochastic autoencoder distributions
\newcommand{\pencode}{p_{\rm{encoder}}}
\newcommand{\pdecode}{p_{\rm{decoder}}}
\newcommand{\precons}{p_{\rm{reconstruct}}}

\newcommand{\laplace}{\mathrm{Laplace}} % Laplace distribution

\newcommand{\E}{\mathbb{E}}
\newcommand{\Ls}{\mathcal{L}}
\newcommand{\R}{\mathbb{R}}
\newcommand{\emp}{\tilde{p}}
\newcommand{\lr}{\alpha}
\newcommand{\reg}{\lambda}
\newcommand{\rect}{\mathrm{rectifier}}
\newcommand{\softmax}{\mathrm{softmax}}
\newcommand{\sigmoid}{\sigma}
\newcommand{\softplus}{\zeta}
\newcommand{\KL}{D_{\mathrm{KL}}}
\newcommand{\Var}{\mathrm{Var}}
\newcommand{\standarderror}{\mathrm{SE}}
\newcommand{\Cov}{\mathrm{Cov}}
% Wolfram Mathworld says $L^2$ is for function spaces and $\ell^2$ is for vectors
% But then they seem to use $L^2$ for vectors throughout the site, and so does
% wikipedia.
\newcommand{\normlzero}{L^0}
\newcommand{\normlone}{L^1}
\newcommand{\normltwo}{L^2}
\newcommand{\normlp}{L^p}
\newcommand{\normmax}{L^\infty}

\newcommand{\parents}{Pa} % See usage in notation.tex. Chosen to match Daphne's book.

\DeclareMathOperator*{\argmax}{arg\,max}
\DeclareMathOperator*{\argmin}{arg\,min}

\DeclareMathOperator{\sign}{sign}
\DeclareMathOperator{\Tr}{Tr}
\let\ab\allowbreak


\usepackage{hyperref}
\usepackage{url}
\usepackage{amsmath}
\usepackage{nicefrac}
\usepackage{comment}
\excludecomment{confidential}
\usepackage{amsmath, scalerel} % assumes amsmath package installed
\usepackage{amssymb}
\usepackage{amsopn}
\usepackage{graphicx}
\usepackage{multirow}

\newcommand{\specialcell}[2][c]{%
	\begin{tabular}[#1]{@{}c@{}}#2\end{tabular}}

\title{Deep Generative Models for learning Coherent Latent Representations from Multi-Modal Data}

% Authors must not appear in the submitted version. They should be hidden
% as long as the \iclrfinalcopy macro remains commented out below.
% Non-anonymous submissions will be rejected without review.

\author{Timo Korthals, Marc Hesse \& Ulrich Rückert\\
Bielefeld University\\
Cluster of Excellence Cognitive Interaction Technologies\\
Cognitronics \& Sensor Systems\\
Inspiration 1, 33619 Bielefeld, Germany\\
\texttt{\{tkorthals,mhesse,rueckert\}@cit-ec.uni-bielefeld.de} \\
\And
J\"urgen Leitner\\
Australian Centre for Robotic Vision\\
Queensland University of Technology\\
Brisbane, Australia\\
\texttt{j.leitner@qut.edu.au} \\
}

% The \author macro works with any number of authors. There are two commands
% used to separate the names and addresses of multiple authors: \And and \AND.
%
% Using \And between authors leaves it to \LaTeX{} to determine where to break
% the lines. Using \AND forces a linebreak at that point. So, if \LaTeX{}
% puts 3 of 4 authors names on the first line, and the last on the second
% line, try using \AND instead of \And before the third author name.

\newcommand{\fix}{\marginpar{FIX}}
\newcommand{\new}{\marginpar{NEW}}

\renewcommand{\a}{a}
\renewcommand{\b}{b}
\renewcommand{\c}{c}
\newcommand{\A}{A}
\newcommand{\B}{B}
\newcommand{\C}{C}
\newcommand{\texta}{\text{a}}
\newcommand{\textb}{\text{b}}
\newcommand{\textc}{\text{c}}
\DeclareMathOperator\dkl{D_{KL}}
\DeclareMathOperator\VarInfo{VI}
\DeclareMathOperator\Info{I}
\DeclareMathOperator\Ent{H}
%\DeclareMathOperator\q{q}
%\DeclareMathOperator\p{p}
\newcommand{\p}{p}
\newcommand{\q}{q}
\newcommand{\elbo}{\mathcal{L}}%{L^{\text{V}}}
\newcommand{\elboJ}{\mathcal{L}_{\text{J}}}%{L^{\text{V}}}
\newcommand{\elboTJ}{\mathcal{L}_{\widetilde{\text{J}}}}%{L^{\text{V}}}
\newcommand{\elboM}{\mathcal{L}_{\text{M}}}
\newcommand{\elboMa}{\elbo_{\text{M}_\texta}}
\newcommand{\elboMb}{\elbo_{\text{M}_\textb}}
\newcommand{\elboMc}{\elbo_{\text{M}_\textc}}
\newcommand{\elboTM}{\elbo_{\widetilde{\text{M}}}}
\newcommand{\elboTMc}{\elbo_{\widetilde{\text{M}}_\textc}}
\newcommand{\elboTMa}{\elbo_{\widetilde{\text{M}}_\texta}}
\newcommand{\elboTMb}{\elbo_{\widetilde{\text{M}}_\textb}}
\newcommand{\optext}[1]{\myspecialcell{\hfill\text{#1}}}
\newcommand{\taref}[1]{\optext{\autoref{#1}}}
\renewcommand{\(}{\mathopen{}\left(}
\renewcommand{\)}{\right)\mathclose{}}
\newcommand{\leftEmptyBrace}{\mathopen{}\left.}
\newcommand{\rightEmptyBrace}{\right.\mathclose{}}
\renewcommand{\[}{\mathopen{}\left[}
\renewcommand{\]}{\right]\mathclose{}}
\DeclareMathOperator{\EX}{\mathbb{E}}% expected value
\DeclareMathOperator{\funf}{f}
\DeclareMathOperator{\fung}{g}
\DeclareMathOperator{\diffdoperator}{d}
\newcommand{\diffd}{\diffdoperator\!}
\DeclareMathOperator{\t2v}{t2v}
\DeclareMathOperator{\F}{F}
\usepackage{icomma}

%\iclrfinalcopy % Uncomment for camera-ready version, but NOT for submission.
\begin{document}
\iclrfinalcopy

\maketitle

\begin{abstract}
The application of multi-modal generative models by means of a \textit{Variational Auto Encoder} (VAE) is an upcoming research topic for sensor fusion and bi-directional modality exchange.
%
This contribution gives insights into the learned joint latent representation and shows that expressiveness and coherence are decisive properties for multi-modal datasets.
%
Furthermore, we propose a multi-modal VAE derived from the full joint marginal log-likelihood that is able to learn the most meaningful representation for ambiguous observations.
%
Since the properties of multi-modal sensor setups are essential for our approach but hardly available, we also propose a technique to generate correlated datasets from uni-modal ones.

\end{abstract}

%\section{INTRODUCTION}
%
%
%However, while the automated discovery of the structure in raw data by a DNN is a time-consuming and error-prone task, an additional pre-processing step that performs feature extraction on the data helps to make DNNs more robust, and to downsize and even transfer them more easily \cite{Higgins2017}.
\textit{Auto Encoder} (AE), \textit{Variational Auto Encoder} (VAE), and more recently \textit{Disentangled Variational Auto Encoder} ($\beta$-VAE) have a considerable impact on the field of data-driven leaning of generative models.
%
Furthermore, recent investigations have shown the fruitful applicability to \textit{deep reinforcement learning} (DRL) as well as bi-directionally exchange of multi-modal data.
%
VAEs tend to encode the data into latent space features that are (ideally) linearly separable as shown by \cite{Higgins2017_2}.
They also allow the discovery of generative joint models (e.g. \cite{Suzuki2017}), as well as zero-shot domain transfer in DRL as shown by \cite{Higgins2017}.
%

However, a good generative model should not just generate good data and achieve a good quantitative score, but also gives a coherent and expressive latent space representation.
%
This property is decisive for multi-modal approaches if the data shows correlation, as it is the case for every sensor setup designed for sensor fusion.
%
With this contribution, we investigate the characteristic of the latent space as well as the quantitative features for existing multi-modal VAEs.
%
Furthermore, we propose a novel approach to build and train a novel multi-modal VAE (M\textsuperscript{2}VAE) which comprises the complete marginal joint log-likelihood without simplifying assumptions.
%
As our objective is the consideration of raw multi-modal sensor data, we also propose an approach to generate correlated multi-modal datasets from available uni-modal ones.
%
Lastly, we draw connections to in-place sensor fusion and epistemic (ambiguity-resolving) active-sensing.

Section \ref{sec:related_work} comprises the related work on multi-modal VAEs.
%
Our comprehensive approach (i.e. M\textsuperscript{2}VAE) is given in Sec. \ref{sec:models}.
%
Furthermore, we describe multi-modal datasets as well as the generation of correlated sets in Sec. \ref{sec:data_generation} which are evaluated in Sec. \ref{sec:experiment}.
%
Finally, we conclude our work in Sec. \ref{sec:conclusion}.



\begin{confidential}
\textbf{-------------------------------------------------}

indicate that only the  

Second, a multi-modal generative model is applied to retrieve a common latent embedding of the observations and to estimate the \textit{evidence lower bound} (ELBO) as a quantity of free energy.


by means of a coherent representation between 
However, while multi-modal VAE approaches do exist, comprehensive evaluation on 

Section \ref{sec:models} presents the derivation of a multi-modal VAE (M²VAE) which suits our needs to find a coherent posterior distribution.




To address this issue, we propose a novel approach to AS which intrinsically reduces ambiguities of observations through epistemic (ambiguity resolving) actions. %goal directed behavior.
%
Friston et al. \cite{Friston2010} state that actions enables to realize preferred outcomes, based on the assumption that both action and perception are trying to maximize the evidence or marginal likelihood of a generative model, as scored by variational free energy.
% Friston \cite{Friston2010} states that by minimizing free energy is equivalent to maximizing model evidence, which is equivalent to minimizing the complexity of accurate explanations for observed outcomes.
Following this principle, if one could directly obtain an estimation of free energy through the current observation, it would enable intrinsically motivated training of autonomous agents to gather information about their environment.
%to resolve ambiguities by their specific modality.
%
Moreover, the agent would learn an epistemic goal-directed behavior, as it would only take the effort of driving to a particular vantage point, iff its sensor modality helps to resolve ambiguity.

%as feature extractor with the intention of exploiting the advertised 
%multi-modal variational Auto Encoder
%
We realize this approach by first downsample the raw sensor observations by means of feature extractors $f_{*}$ as shown if Fig. \ref{fig:arch}. 
%
Second, a multi-modal generative model is applied to retrieve a common latent embedding of the observations and to estimate the \textit{evidence lower bound} (ELBO) as a quantity of free energy.
%
We achieved this by deriving an objective from the joint marginal likelihood that reveals to an unsupervised training of a coherent posterior distribution between all modality permutations by means of a variational Auto Encoder (VAE).
%
Third, we train an agent applying a deep reinforcement learning (DRL) approach on the latent embedding of the VAE with the VAE's ELBO estimations as reward signal to perform epistemic actions wrt. its modality.

%to derive an objective for a multi-modal variational auto encoder, that 
%Furthermore, 
%, given partial observations of the 
%
%by the model to derive reasonable navigation goals that minimize expected free energy \cite{friston2016} in a multi-agent active sensing setup.
%
%The overall goal of this work is the investigation of reinforcement approaches for distributed sensing based on latent space representations derived from multi-modal deep generative models.
%The main objective of this contribution is to train multi-modal VAE that integrates all the information on different sensor modalities into a joint latent representation.
%and then to generate one sensor information from the corresponding other one via this joint representation.
%Therefore, this model can exchange multiple sensor modalities bi-directionally, for example, features from laser scanner data to images and vice versa, and can learn a shared latent space distribution between uni- and multi-modal cases.
%Furthermore, we train a deep Q-Network that controls robots equipped with uni-modal sensors directly on the latent space with the objective of reducing uncertainty regarding detected objects. Our approach performs better than naive multi-robot exploration.
%
%\textbf{----------------------------------------------------}
%The overall goal of this project is the investigation of reinforcement approaches for distributed sensing based on inverse sensor models.
%on latent space representations derived from multi-modal deep generative models.
%
%The main objective of this contribution is to train a multi-modal VAE that integrates all information on different sensor modalities into a joint latent representation.
%
%We follow the approach of applying an unsupervised trained multi-modal VAE as feature extractor with the intention of exploiting the advertised \textit{evidence lower bound} (ELBO) by the model to derive reasonable navigation goals that minimize expected free energy \cite{friston2016} in a multi-agent active sensing setup.
%
%The proposed architecture, as shown in Fig. \ref{fig:arch} comprises three stages as: 1) learning to sense, 2) learning to combine, 3) learning to act.
%
%The goal of an actor is then, to select navigation goals in the environment by selecting sensor modalities in a manner that a perception minimizes the free energy which is equivalent to, if formulated reciprocally, the maximization of future expected reward.
%
%\begin{figure}[thpb]
%	% \small
%	\footnotesize
%	%\input{architecture_mmvae_back_connection.pdf_tex}
%	\input{architecture_mmvae_back_connection_decoder.pdf_tex}
%	% \input{architecture_RL_application.pdf_tex}
%	%\caption{Active sensing architecture of a single agent comprising three stages: 1) features are extracted from a subset of sensor modalities $f_{*}$ or from the M²VAE decoder networks; 2) a common state representation of the features is generated and mapped by the M²VAE encoders, 3) an actor network chooses the next sensor modality, such that the ELBO (i.e. quantity of free energy) is minimized.}
%    \caption{Proposed architecture of a single agent comprising: 1) downsampling and feature extraction; 2) state and reward estimation by the M²VAE; 3) epistemic action selection of the next vantage point for its modality to minimize free energy.}
%	\label{fig:arch}
%\end{figure}
%
%\textbf{----------------------------------}
%
%In general, selecting actions based upon value of states only works when the states are known.
%\textbf{--------------FRISTON START--------------------}
%
%In the case of ambiguous or partially observations, an uncertainty mapping between hidden states and outcomes is necessary.
%
%This means that acting to minimize free energy resolves ambiguity and unsurprising or preferred outcomes.
%
%are made 
%\textbf{--------------FRISTON END--------------------}
%
%Our contribution makes use of the fields of deep neuronal networks for feature extraction, deep generative models for latent representations, and deep Q-Networks for optimal control in heterogeneous multi-agent systems in order to archive sufficient classification of objects in the environment.
Since this contribution concentrates on the generative models as the central feature enabling our approach, we stress this part in the Sec. \ref{sec:related_work}.
%
Section \ref{sec:models} presents the derivation of a multi-modal VAE (M²VAE) which suits our needs to find a coherent posterior distribution.
%
Furthermore, we argue that the features of the M²VAE can be applied as state information as well as reward signal for a DRL approach in Sec. \ref{sec:RL}.
%
An evaluation wrt. to three multi-modal VAEs is done in Sec. \ref{sec:experiment} with the application to a \textit{multi-agent DRL} (MARL), ambiguity resolving, AS scenario.
%
Lastly, we conclude our work in Sec. \ref{sec:conclusion}.
%\autoref{seq:RL} provides an overview of our application within RL, which is then evaluated in \autoref{seq:setup}.
%gives the overview and application of the trained latent space to the MARL exploration task.


\end{confidential}

%\section{RELATED WORK}
\label{sec:related_work}


%First, we want to give a brief review of VAEs \cite{Kingma2014} which constitute the basis of the multi-modal case. %proposed by Suzuki et al. \cite{Suzuki2017}.
\textit{Variational auto encoder} (VAE) combine neural networks with variational inference to allow unsupervised learning of complicated distributions according to the graphical model shown in \autoref{fig:VAE} (left).
%
A $D_\a$-dimensional observation $\a$ is modeled in terms of a $D_z$-dimensional latent vector $z$ using a probabilistic decoder $\p_{\theta_\texta}\(z\)$ with parameters $\theta$.
%
%$\p_{\theta_\texta}\(z\)$ is given by a neural network which is referred to as probabilistic decoder as it provides the mean of our observation model.
%
%The vector $z$ can be seen as a code of the corresponding observation, which is assumed to be drawn from a standard normal distribution $z\!\sim\!p(z)\!=\!\mathcal(z; 0, \mathbf{I})$.
%
%The choice of this distribution is justified since it can be transformed to an arbitrary distribution given a sufficiently complex transformation $\p_{\theta_\texta}\(z\)$.
%
To generate the corresponding embedding $z$ from observation $\a$, a probabilistic encoder network with $\q_{\phi_\texta}\(z\)$ is being provided which parametrizes the posterior distribution from which $z$ is sampled.
%
The encoder and decoder, given by neural networks, are trained jointly to bring $\a$ close to an $\a'$ under the constraint that an approximate distribution needs to be close to a prior $\p\(z\)$ and hence inference is basically learned during training.

The specific objective of VAEs is the maximization of the marginal distribution $\p\(\a\) = \int \p_{\theta}\(\a|z\)\p\(z\)\diffd\a$.
Because this distribution is intractable, the model is instead trained via \textit{stochastic gradient variational Bayes} (SGVB) by maximizing the \textit{evidence lower bound} (ELBO) $\elbo$ of the marginal log-likelihood $\log\p\(\a\):=L_{\texta}$ as
\begin{align}
L_{\texta}\! \geq\! \elbo\! =\! \underbrace{- \dkl \( \q_\phi \( z| \a \) \| \p\( z \) \)}_\text{Regularization}
+ 
\underbrace{\EX_{\q_\phi\( z | \a \)} \log \( \p_\theta \( \a | z \)  \)}_{\text{Reconstruction}}\text{.}
\label{eq:reg_rec}
\end{align}
%The VAE model is used in settings where only a single modality $x$ is present, in order to find a latent encoding $z$ (c.f. \autoref{fig:VAE} right).
This approach proposed by \cite{DBLP:journals/corr/KingmaW13} is used in settings where only a single modality $\a$ is present in order to find a latent encoding $z$ (c.f. \autoref{fig:VAE} (left)).

%
In the following chapters, we give a briefly comprise related work by means of multi-modal VAEs.
%
Further, we stress the concept of two joint multi-modal approaches to derive the later proposed  \textit{variational Auto Encoder} (VAE).
% and the extension to multiple input modalities.
%Further, we give a brief overview of applications where sensor data pre-processing techniques, in particular VAEs, have been exploited as feature extractors to boost deep reinforcement learning. 
\begin{figure}
	\footnotesize
	\begin{center}
		%\input{VAEs_v2.pdf_tex}
		%\input{VAEs.pdf_tex}
		\input{VAEs_trimodal.pdf_tex}
	\end{center}
	\caption{Evolution of full uni-, bi-, and tri-modal VAEs comprising all modality permutations}
	\label{fig:VAE}
\end{figure}
%
\subsection{Multi-Modal Auto Encoder}
\label{sec:vae}
%
Given a set of modalities $\mathcal{M}\!=\!\left\lbrace\a,\b,\c,\ldots\right\rbrace$, multi-modal variants of \textit{Variational Auto Encoder}s (VAE) have been applied to train generative models for multi-directional reconstruction (i.e. generation of missing data) or feature extraction.
%
Variants are \textit{conditional VAEs} (CVAE) and conditional multi-modal autoencoders (CMMA), with the lack in bi-directional reconstruction (\cite{NIPS2015_5775,Pandey2016}).
%
BiVCCA by \cite{Wang2016_2} trains two VAEs together with interacting inference networks to facilitate two-way reconstruction with the lack of directly modeling the joint distribution. %, which we find empirically to improve the ability of a model to learn the data distribution.
%
Models, that are derived from the \textit{variation of information} (VI) with the objective to estimate the joint distribution with the capabilities of multi-directional reconstruction were recently introduced by \cite{Suzuki2017}.
%
\cite{Vedantam2017} introduce another objective for the bi-modal VAE, which they call the triplet ELBO (tVAE).
%
%First, the full multi-modal VAE is trained, after which the encoder weights are pinned to train the remaining uni-modal networks.
%
%Wu et al. \cite{Wu2018} propose a Product-of-Expert architecture combining the variational distribution, which has to be Gaussian, of the set of all uni-modal encoders
%More work on multi-modal (AE) which are used as feature extractors are
%
Furthermore, multi-modal stacked Auto Encoders (AE) are a variant of combining the latent spaces of various AEs ( \cite{Larochelle:2007:EED:1273496.1273556,Ranzato:2006:ELS:2976456.2976599}) which can also be applied to the reconstruction of missing modalities (\cite{Ngiam2011,Cadena}).
%
% That means, that inputs may have zero values for one of the input modalities and original values for the other input modality, but still require the network to reconstruct both modalities.
%
However, while \cite{Suzuki2017} and \cite{Vedantam2017} argue that training of the full multi-modal VAE is intractable, because of the $2^{|\mathcal{M}|}\!-\!1$ modality subsets of inference networks, we show that training the full joint model estimates the most expressive latent embeddings.

\subsubsection{Joint Multi-Modal Variational Auto Encoder}
When more than one modality is available, e.g. $\a$ and $\b$ as shown in \autoref{fig:VAE} (mid.), the derivation of the ELBO $\elboJ$ for a marginal joint log-likelihood $\log\p\(\a\):=L_{\text{J}}$ is straight forward:
\begin{align}
&L_{\text{J}} \geq \elboJ =
\underbrace{- \dkl \( \q_{\phi_{\texta\textb}} \( z| \a,\b \) \| \p\( z \) \)}_{\text{Regularization}}
+
\underbrace{\EX_{\q_{\phi_{\texta\textb}}\( z | \a,\b \)} \log \( \p_{\theta_\texta} \( \a | z \)  \)}_{\text{Reconstruction wrt. $\a$}}
+
\underbrace{\EX_{\q_{\phi_{\texta\textb}}\( z | \a,\b \)} \log \( \p_{\theta_\textb} \( \b | z \)  \)}_{\text{Reconstruction wrt. $\b$}} \label{eq:joint_reg_rec_rec}
\end{align}
However, it is not clear how to perform inference if the dataset consists of samples lacking from modalities (e.g. for samples $i$ and $k$: $\(a_i,\varnothing\)$ and $\(\varnothing,b_k\)$).
\cite{Ngiam2011} propose training of a bimodal deep auto encoder using an augmented dataset with additional examples that have only a single-modality as input.
We, therefore, name the resulting model of Eq. \ref{eq:joint_reg_rec_rec} \textit{joint multi-modal VAE-Zero} (JMVAE-Zero).
%However, it is less clear how to apply or train the uni-modal encoders via this expression and further, how to perform inference if the dataset consists of samples lacking from modalities (e.g. for samples $i$ and $k$: $\(a_i,\varnothing\)$ and $\(\varnothing,b_k\)$).
%
\subsubsection{Joint Multi-Modal Variational Auto Encoder from Variation of Information}
%However, if missing modalities are high-dimensional and complicated such as natural images, then the inferred latent variable becomes incomplete and generated samples might collapse.
While the former approach cannot directly be applied to missing modalities, \cite{Suzuki2017} propose a \textit{joint multi-modal VAE} (JMVAE) that is trained via two uni-modal encoders and a bi-modal en-/decoder which share one objective function derived from the \textit{variation of information} (VI) of the marginal conditional log-likelihoods $\log \p\(\a|\b\)\p\(\b|\a\)=:L_{\text{M}}$ by optimizing the ELBO $\elboM$:
\begin{align}
&L_{\text{M}} \geq \elboM \geq \elboJ - \underbrace{\dkl \( \q_{\phi_{\texta\textb}} \( z | \a,\b \) \| \q_{\phi_{\textb}}\( z| \b \) \)}_{\text{Unimodal PDF fitting of encoder b}} - \underbrace{\dkl \( \q_{\phi_{\texta\textb}} \( z | \a,\b \) \| \q_{\phi_{\texta}}\( z| \a \) \)}_{\text{Unimodal PDF fitting of encoder a}}
%\label{eq:multi_reg_rec}
\end{align}
%Therefore, Suzuki et al. \cite{Suzuki2017} propose a joint multi-modal VAE (JMVAE) that is trained on two uni-modal encoder and bi-modal en-/decoders which share an objective function derived from the \textit{variation of information} (VI) between $\a$ and $\b$.
Therefore, uni-modal encoders are trained, so that their distributions $q_{\phi_{\a}}$ and $q_{\phi_{\b}}$ are close to a multi-modal encoder $q_{\phi_{\texta\textb}}$ in order to build a coherent posterior distribution.
%
The introduced regularization by \cite{Suzuki2017} puts learning pressure on the uni-modal encoders just by the distributions' shape, disregarding reconstruction capabilities and the prior $\p\(z\)$.
%
Furthermore, one can show that deriving the ELBO from the VI for a set of $\mathcal{M}$ observable modalities, always leads to an expression of the ELBO that allows only training of $\widetilde{\mathcal{M}}=\left\lbrace m | m \in \mathcal{P}\( \mathcal{M}\), |m|=|\mathcal{M}|-1 \right\rbrace$ modality combinations.
%
This leads to the fact that for instance in a tri-modal setup, as shown in Fig. \ref{fig:VAE} (right), one can derive three bi-modal encoders from the VI, but no uni-modal ones.




%\section{Multi-Modal Variational Auto Encoder Approach}
\label{sec:models}
\label{sec:wgm}
%
While the objective of \cite{Wang2016_2}, \cite{Ngiam2011}, \cite{Suzuki2017}, and \cite{Vedantam2017} is to exchange modalities bi-directionally (e.g. $\a\rightarrow\b'$), our primary concern is twofold:
%
First, find a meaningful posterior distribution where the sampled statistics of an encoder network allows inference about further actions.
%
Second, find an expression to jointly train all $2^{|\mathcal{M}|}\!-\!1$ permutations of modality encoders.

%
By successively applying logarithm and Bayes rules, we derive the ELBO for the multi-modal VAE (M\textsuperscript{2}VAE) as follows:
%
First, given the independent set of observable modalities $\mathcal{M}=\lbrace\a,\b,\c,\ldots\rbrace$, its marginal log-likelihood $\log \p\(\mathcal{M}\)=:L_{\text{M\textsuperscript{2}}}$ is multiplied by the cardinality of the set as the neutral element $1=\nicefrac{|\mathcal{M}|}{|\mathcal{M}|}$.
%
Second, applying logarithm multiplication rule, the nominator is written as the argument's exponent.
%
Third, Bayes rule is applied to each term wrt. the remaining observable modalities to derive their conditionals.
%
Further, we bootstrap the derivation technique in a bi- and tri-modal (c.f. tri-modal case in Sec. \ref{seq:extension_three_mod_suzuki}) case to illustrate the advantages.
%
By excessively applying the scheme until convergence of the mathematical expression, it leads for a bi-modal set $\mathcal{M}=\lbrace\a,\b\rbrace$ to the following result:
\begin{align}
L_{\text{M\textsuperscript{2}}} &= \nicefrac{2}{2}\log\p\( \a,\b\) = \nicefrac{1}{2}\log\p\( \a,\b\)^2 = \nicefrac{1}{2}\log\p\( \a,\b\)\p\( \a,\b\) = \nicefrac{1}{2}\log\p\( \b \)\p\( \a|\b\)\p\( \b|\a\)\p\( \a \) \\
              &= \nicefrac{1}{2}\( \log\p\( \a \) +  \log\p\( \b|\a\) + \log\p\( \a|\b\) + \log\p\( \b \)  \) = \nicefrac{1}{2}\( L_{\texta} +  L_{\text{M}} + L_{\textb}  \)
%              &\geq \nicefrac{1}{2}\(\elbo_{\text{a}} + \elboMa + \elboMb + \elbo_{\text{b}}\) \\
%              &\geq \nicefrac{1}{2}\(\elbo_{\text{a}} + \elboM + \elbo_{\text{b}}\)
\end{align}
%

This term can be written as inequality wrt. each ELBO of the marginals $L_{\texta}$, $L_{\textb}$ and conditionals $L_{\text{M}}$:
\begin{align}
&2 L_{\text{M\textsuperscript{2}}} \geq 2 \elbo_{\text{M\textsuperscript{2}}} = \elbo_{\text{a}} + \elbo_{\text{b}} + \elboM = \label{eq:m2_elbo_sum}\\
&- \beta_{\texta} \dkl \( \q_{\phi_{\texta}} \( z| \a \) \| \p\( z \) \) + \EX_{\q_{\phi_{\texta}}\( z | \a \)} \log \( \p_{\theta_{\texta}} \( \a | z \)  \) \label{eq:m2_elbo_a}\\
& - \beta_{\textb} \dkl \( \q_{\phi_{\textb}} \( z| \b \) \| \p\( z \) \) + \EX_{\q_{\phi_{b}}\( z | \b \)} \log \( \p_{\theta_{\textb}} \( \b | z \)  \) \label{eq:m2_elbo_b}\\
&+ \EX_{\q_{\phi_{\texta\textb}}\( z | \a,\b \)} \log \( \p_{\theta_{\texta}} \( \a | z \)  \) + \EX_{\q_{\phi_{\texta\textb}}\( z | \a,\b \)} \log \( \p_{\theta_{\textb}} \( \b | z \)  \) - \beta_{\texta\textb}\dkl \( \q_{\phi_{\texta\textb}} \( z| \a,\b \) \| \p\( z \) \) \label{eq:m2_elbo_jmvae}\\
& - \alpha\dkl \( \q_{\phi_{\texta\textb}} \( z | \a,\b \) \| \q_{\phi_{\texta}}\( z| \a \) \) - \alpha\dkl \( \q_{\phi_{\texta\textb}} \( z | \a,\b \) \| \q_{\phi_{\textb}}\( z| \b \) \) \label{eq:m2_elbo_jmvae_mutual_ab}
\text{.}
\end{align}
%
%\begin{align}
%&2 L_{\text{M\textsuperscript{2}}} \geq 2 \elbo_{\text{M\textsuperscript{2}}} = \elbo_{\text{a}} + \elbo_{\text{b}} + \elboM = \label{eq:m2_elbo_sum}\\
%&- \beta_{\texta} \dkl \( \q_{\phi_{\texta}} \( z| \a \) \| \p\( z \) \) + \EX_{\q_{\phi_{\texta}}\( z | \a \)} \log \( \p_{\theta_{\texta}} \( \a | z \)  \) \label{eq:m2_elbo_a}\\
%& - \beta_{\textb} \dkl \( \q_{\phi_{\textb}} \( z| \b \) \| \p\( z \) \) + \EX_{\q_{\phi_{b}}\( z | \b \)} \log \( \p_{\theta_{\textb}} \( \b | z \)  \) \label{eq:m2_elbo_b}\\
%&+ \EX_{\q_{\phi_{\texta\textb}}\( z | \a,\b \)} \log \( \p_{\theta_{\texta}} \( \a | z \)  \) + \EX_{\q_{\phi_{\texta\textb}}\( z | \a,\b \)} \log \( \p_{\theta_{\textb}} \( \b | z \)  \) \label{eq:m2_elbo_jmvae_rec}\\
%& - \beta_{\texta\textb}\dkl \( \q_{\phi_{\texta\textb}} \( z| \a,\b \) \| \p\( z \) \) \label{eq:m2_elbo_jmvae_dkl}\\
%& - \alpha\dkl \( \q_{\phi_{\texta\textb}} \( z | \a,\b \) \| \q_{\phi_{\texta}}\( z| \a \) \) \label{eq:m2_elbo_jmvae_mutual_a}\\
%& - \alpha\dkl \( \q_{\phi_{\texta\textb}} \( z | \a,\b \) \| \q_{\phi_{\textb}}\( z| \b \) \) \label{eq:m2_elbo_jmvae_mutual_b}
%\text{.}
%\end{align}
%
\autoref{eq:m2_elbo_sum} is substituted by all formerly derived ELBO expressions lead to the combination of the uni-modal VAEs wrt. $\texta$ and $\textb$ (c.f. Eq. \ref{eq:m2_elbo_a} to \ref{eq:m2_elbo_b}) and the JMVAE comprising the VAE wrt. the joint modality $\texta\textb$ (c.f. Eq. \ref{eq:m2_elbo_jmvae}) and mutual latent space (c.f. Eq. \ref{eq:m2_elbo_jmvae_mutual_ab}). 
%
\autoref{eq:m2_elbo_a} and \ref{eq:m2_elbo_b} have the effect that their regularizers care about the uni-modal distribution to deviate not too much from the common prior while their reconstruction term shapes the underlying embedding of the mutual latent space.
%
We further apply the concept of $\beta$-VAE (\cite{Higgins2016,Higgins2017_2,Burgess2018}) to the regularizers via $\beta_{*}$ and adopt the factor $\alpha$ from \cite{Suzuki2017} for the mutual regularizer.
%
However, while $\beta$-VAE have the property to disentangle the latent space, our main concern is the balance between the input and the latent space using a constant normalized factor $\beta_{\text{norm}}=\beta_{*} \nicefrac{D_{*}}{D_{z}}$.
% \approx 10^{-2} \ldots 10^{-3}

If the derivation, which we leave out for the sake of brevity, is applied to the log-likelihood $L_{\text{M\textsuperscript{2}}_{\mathcal{M}}}$ of a set $\mathcal{M}$, one can show that it results into a recursive form consisting of JMVAEs' and M\textsuperscript{2}VAEs' log-likelihood terms
%\begin{align}
%L_{\text{M\textsuperscript{2}}_{\mathcal{M}}} &= \frac{1}{\(|\mathcal{M}|^2 -|\mathcal{M}|\)} \sum_{\widetilde{m}\in\widetilde{\mathcal{M}}} L_{\text{M\textsuperscript{2}}_{\widetilde{m}}} + \frac{1}{|\mathcal{M}|} L_{\text{M}_\mathcal{M}}\label{eq:log_expression_induction}\\
%%L_{_{|\mathcal{M}|}\text{M\textsuperscript{2}}_{\mathcal{M}}} = \frac{1}{\(|\mathcal{M}|^2 -|\mathcal{M}|\)} \sum_{\widetilde{m}\in\widetilde{\mathcal{M}}} \! L_{_{|\widetilde{m}|}\text{M\textsuperscript{2}}} + \frac{1}{|\mathcal{M}|} L_{_{|\mathcal{M}|}\text{M}_\mathcal{M}}
%&\geq \frac{1}{\(|\mathcal{M}|^2 -|\mathcal{M}|\)} \sum_{\widetilde{m}\in\widetilde{\mathcal{M}}} \elbo_{\text{M\textsuperscript{2}}_{\widetilde{m}}} + \frac{1}{|\mathcal{M}|} \elbo_{\text{M}_\mathcal{M}} \label{eq:log_expression} \\
%&=: \elbo_{\text{M\textsuperscript{2}}_{\mathcal{M}}}\text{.}
%\end{align}
%\begin{align}
%L_{\text{M\textsuperscript{2}}_{\mathcal{M}}} &= \frac{1}{|\mathcal{M}|} \( L_{\text{M}_{\mathcal{M}}} + \frac{1}{\(|\mathcal{M}| - 1\)}  \( \sum_{\widetilde{m}\in\widetilde{\mathcal{M}}} L_{\text{M}_{\widetilde{m}}} + 2\sum_{\widetilde{\widetilde{m}}\in\widetilde{\widetilde{\mathcal{M}}}} L_{\text{M\textsuperscript{2}}_{\widetilde{\widetilde{m}}}} \)\) &\label{eq:log_expression_induction}\\
%%
%&\geq \frac{1}{|\mathcal{M}|} \( \elbo_{\text{M}_{\mathcal{M}}} + \frac{1}{\(|\mathcal{M}| - 1\)}  \( \sum_{\widetilde{m}\in\widetilde{\mathcal{M}}} \elbo_{\text{M}_{\widetilde{m}}} + 2\sum_{\widetilde{\widetilde{m}}\in\widetilde{\widetilde{\mathcal{M}}}} \elbo_{\text{M\textsuperscript{2}}_{\widetilde{\widetilde{m}}}} \)\)
%%&\geq \frac{1}{\(|\mathcal{M}|^2 -|\mathcal{M}|\)} \sum_{\widetilde{m}\in\widetilde{\mathcal{M}}} \elbo_{\text{M\textsuperscript{2}}_{\widetilde{m}}} + \frac{1}{|\mathcal{M}|} \elbo_{\text{M}_\mathcal{M}} \label{eq:log_expression} \\
%&=: \elbo_{\text{M\textsuperscript{2}}_{\mathcal{M}}}\text{.}
%\end{align}
%
\begin{align}
L_{\text{M\textsuperscript{2}}_{\mathcal{M}}} = \frac{1}{|\mathcal{M}|} \( L_{\text{M}_{\mathcal{M}}} + \sum_{\widetilde{m}\in\widetilde{\mathcal{M}}} L_{\text{M\textsuperscript{2}}_{\widetilde{m}}} \)
%
\geq \frac{1}{|\mathcal{M}|} \( \elbo_{\text{M}_{\mathcal{M}}} + \sum_{\widetilde{m}\in\widetilde{\mathcal{M}}} \elbo_{\text{M\textsuperscript{2}}_{\widetilde{m}}} \)
%
=: \elbo_{\text{M\textsuperscript{2}}_{\mathcal{M}}}\text{.} \label{eq:log_expression_induction}
\end{align}


%While Eq. \ref{eq:log_expression_induction} can be proven via induction, which we leave out for the sake of brevity.
%
%with $\widetilde{\widetilde{\mathcal{M}}}=\left\lbrace m | m \in \mathcal{P}\( \mathcal{M}\), |m|=|\mathcal{M}|-2 \right\rbrace$.
%
While the derivation of Eq. \ref{eq:log_expression_induction} is given in Sec \ref{sec:proof_mmvae}, the properties are as follows:
\begin{itemize}
	\item the M\textsuperscript{2}VAE consist out of $2^{|\mathcal{M}|}\!-\!1$ encoders and $|\mathcal{M}|$ decoders comprising all modality combinations
	%(we therefore refer M\textsuperscript{2}VAE to be named \textit{wide generative model} (WGM))
	\item while it also allows the bi-directional exchange of modalities, it further allows the setup of arbitrary modality combinations having $1$ to $|\mathcal{M}|$ modalities
	\item subsets of minor cardinality are weighted less and have a therefore minor impact in shaping the overall posterior distribution (vice versa, the major subsets dominate the shaping and the minor sets adapt to it)
	\item all encoder/decoder networks can jointly be trained using SGVB
\end{itemize}
%It consist out of $2^{|\mathcal{M}|}-1$ encoders and $|\mathcal{M}|$ decoders comprising all modality combinations;
%while it also allows the bi-directional exchange of modalities, it further allows the setup of arbitrary modality combinations having $1$ to $|\mathcal{M}|$ modalities; 
%subsets of minor cardinality are weighted less and have therefore minor impact in shaping the overall posterior distribution (vice versa, the major subsets dominate the shaping and the minor sets adapt to it).


%to control the learning pressure 

%and adopt the 
%The expression trains all encoder and decoder networks jointly to maximize



%With the JMVAE, we can extract joint latent features by sampling from the joint encoder $q_{\phi}\(z|\a,\b\)$ at testing time.


%While the objective of \cite{Suzuki2017} is to exchange modalities bi-directionally ($\a\rightarrow\b'$ and $\b\rightarrow\a'$), our primary concern is to find a meaningful posterior distribution and hence, we analyze their approach with a view of using all statistics provided by the encoder network.


\begin{confidential}
Writing as a Lagrangian we obtain the familiar variational free energy objective function shown in Eq. 3 (19, 26), where $\beta$ is the inverse temperature or regularisation coefficient


While learning wrt. the free energy principle is the minimization of surprise we argue that active-sensing aims to find an action with maximum information gain to minimizes future surprises and hence, the number of further necessery actions.

An increase of complexity will always lead to a decrease of free energy iff the same external state is respected.

the increase of accuracy.

So the objective becomes to match the internal states of representation to the external ones by actively choosing sensor modalities that choose.

We extend the free energy term to posses multiple modalities for one single external state.

This is exactly what Bayes inference does when maximizing the evidence lower bound during training: i.e. for the sake of increasing the complexity of latent representation, higher accuracy is archived.
That is analoug of minimizing the free energy, and therefore future surprises, through sensing.

Test of JMVAE



We encode the numbers 1,7,8 into RGB channel and perceive the 
image with our two eyes, while one eye ($a$) is trichromatic (able to distinguish all three color channel) and the other one ($b$) is dichromatic (not able to distinguish between red and blue).
We now try to find an embedding using JMVAE which will exclusively shape the latent space by the joint reconstruction loss resulting in
p(z|x,y): 1111111111777777777788888888
xy: decides between 1,7,8
While blinking blinking with just one eye, the JMVAE only tries to match the latent space distribution onto 7, and the pair 1,8, resulting into the following reconstruction:
p(z|x,y): 7777777777777777777777777777

On the other hand, the M\textsuperscript{2}VAE shapes the latent space also via the uni-modal reconstruction loss, resulting into
p(z|x,y): 7777777777888888881111111111
p(z|x,y): 7777777777818181818181818181


\end{confidential}

\section{Introduction}
\label{sec:data_generation}
%
It is quite common in the multi-modal VAE community to model a bi-modal dataset as follows (\cite{Wang2016_2,Ngiam2011,Suzuki2017,Vedantam2017}):
%
The first modality $\a$ denotes the raw data and $\b$ denotes the label (e.g. the digits' images and labels as one-hot vector wrt. the MNIST dataset).
%
This is a rather artificial assumption and only sufficient when the objective is within a semi-supervised training framework.
%
Real multi-modal data does not show this behavior as there are commonly multiple raw data inputs.
%
Unfortunately, only complex multi-modal datasets of heterogeneous sensor setups exist (\cite{Ofli2013,udacity2016,kragh2017fieldsafe}), which makes a comprehensive evaluation for VAEs futile.
%
On the other hand, creating own multi-modal datasets is exhaustive since  training generative models either demand dense sampling or supervised signals to form a consistent latent manifold (\cite{bengio2012}).
%one either needs a dense data-set or have to create labeled data, because
%
%This ambiguity may be resolved through either dense sampling of the manifolds or by adding supervised signals. 
%
%The importance of vast quantities of unlabeled data for the success of unsupervised approaches in learning disentangled factor representations was pointed by \cite{bengio2012}.

While na\"ive consolidation of non-coherently datasets does not meet the conditions of data continuity, as discussed later, we propose a consolidation technique by sampling from superimposed latent spaces of various uni-modal trained CVAEs in Sec. \ref{sec:mm_set_generation}.
%
This approach allows the generation of multi-modal datasets from distinct and disconnected uni-modal sets.
%

\subsection{MNIST-E}
\label{sec:mm_set_generation}

\cite{Perry2010} state that Hebbian learning relies on the fact that the same objects are continuously transformed to their nearest neighbor in the observable space.
%
\cite{Higgins2016} adopted this approach to their assumptions, that this notion can be generalized within the latent manifold learning.
%
Further, neither a coherent manifold nor a proper factorization of the latent space can be trained if these assumptions are not fulfilled by the dataset.
%
In summary, this means that observed data has to have the property of continues transformation wrt. to their properties (e.g. position and shape of an object), such that a small deviation of the observations results in proportional deviations in the latent space.
%
%, such that a coherent latent manifold can be trained.
% to learn a sufficient factorization in latent space.
% (i.e. is sampled from a )
% Moreover, we propose a technique to match-up existing datasets to meet the conditions of real raw multi-modal datasets.
%
We adopt this assumption for multi-modal datasets where observations should correlate if the same quantity is observed, such that a small deviation in the common latent representation between all modalities conducts a proportional impact in all observations.
%
This becomes an actual fundamental requirement for any multi-modal dataset, as correlation and coherence are within the objective of multi-modal sensor fusion.
%
%These assumptions only hold, iff the modality is able to rectify the target.
%
%This means speaking from the genrative model perspective for two modalities $a$ and $b$, there exists an likelihood $\p\(\a|z_1,z_2 \)$ and $\p\(\b|z_1\)$.
%
%However, in a factorized assumption the likelihoods become $\p\(\a|z_1,z_2\)$ and $\p\(\b|z_1,z_2\)$ and if $\b$ cannot sense a particular property, it is likely that it becomes independent of one factor (i.e. $\p\(\b|z_1\)$).
%
%This means that $\b$ makes ambiguous observations which can only be resolved if the rectification is done by $\a$ or $\a$ and $\b$ together.
%
%Moreover, we extend the postulation by Higgins et al. \cite{Higgins2016}, that it is important that the observed multi-modal data is generated using factors of variation that are densely sampled from their respective continuous distributions.
%
%In that case, and if the posterior is trained under the same prior, we propose that the overlay of latent factors suffice this assumption.
%
%In summary, that means that observed data has to have the property of continues transformation (i.e. is sampled from a ), such that to learn a sufficient factorization in latent space.
%
%Further more, a small change on the manifold of the posterior probability should result in a proportional deviation on the likelihood manifold.
%
%Or more over, a multi-modal observation should correlate if the same object is observed and moreover, the modalities can sense the object
%
%Here we specify a particular aspect of the data we believe is important for such learning.
%
%We postulate that it is important that the observed data is generated using factors of variation that are densely sampled from their respective continuous distributions.
%
In the following, we propose a technique to generate new multi-modal datasets, given different uni-modal enclosed sets which meet the former conditions.

A valuable property of the VAE's learned posterior distribution is, that it matches the desired prior quite sufficiently if only a single class is observed.
%
This characteristic can be found again in the conditional VAE (CVAE) \cite{Kingma2014,NIPS2015_5775} as it's training is supported by the ground truth labels of the observations.
%
Thus, it actually builds non-related posterior distribution for each class label, where every distribution matches a given prior.
%
Furthermore, we adopt the idea of $\beta$-VAE \cite{Higgins2017} which learns disentangled and factorized latent representations.
%
Combining the properties of both advantages allows the superimposing of latent manifolds from various uni-modal encoders as shown in Fig. \ref{fig:e_mnist} (Top-Right).
%
Now, latent samples can be drawn from the posterior to operate all CVAE encoders, with the desired label, to generate continues multi-modal data.
%

To test the approach we consolidate MNIST (\cite{LeCun1998}) and fashion-MNIST (\cite{Xiao2017}) to an entangled \textit{MNIST} (MNIST-E) set by sampling from the prior (i.e. $z\sim\mathcal{N}\( 0,\mathbf{I} \)$) to generate observation tuples from the corresponding encoder networks $\p_{\theta_{\a}}\(\a|z,C\)$ and $\p_{\theta_{\b}}\(\b|z,C\)$ with class label $C$.
%
The network architecture is explained in Sec. \ref{tab:architectures_emnist}.
%
To avoid artifacts, only samples from within $2\sigma$ of the prior are obtained.

Furthermore, we train a bi-modal JMVAE on the newly generated data to depict properties of the different datasets.
%
We are aware of the fact that consolidation of uni-modal datasets cannot be achieved easily since continuity is hardly measurable.
%
Therefore, na\"ive consolidation results in a mixed dataset (i.e. mixed-MNIST) as shown in Fig. \ref{fig:e_mnist}.
%
To mimic this behavior and to achieve a fair comparison of the ELBO, we shuffle the generated fashion-MNIST per class label of MNIST-E to generate an equivalent mixed \textit{MNIST-E} (MNIST-ME) set.
%

As shown in Fig. \ref{fig:e_mnist} (bottom), the JMVAE's latent space reveals that for MNIST-M single clusters share the same mean as the best representative of a single label, but the variance of any uni-modal trained encoder remains orthogonal.
%
Thus, the continuity in the observations does not correlate with each other by any means.
%
On the other hand, the MNIST-E set with continues samples shows the desired behavior of multi-modal datasets as the JMVAE trains a coherent distribution for all uni- and multi-modal encoders.
%
These observations show that our proposed approach for generating new entangled datasets meet the formulated requirements of multi-modal datasets.

%
%
%
%
%This can also be associated, to the necessary information channel which is used by the VAE.
%
%The VAE drives up the KL-divergence to 
%
%
%
%
\begin{figure}[h]
	\footnotesize
	\begin{center}
		%\input{VAEs_v2.pdf_tex}
		%\input{VAEs.pdf_tex}
		\input{MNIST_eval.pdf_tex}
	\end{center}
	\caption{\textbf{Top-Left}: Depiction of na\"ive mixed MNIST (m-MNIST) vs. proposed entangled MNIST (e-MNIST).
		%
		m-MNIST is pairwise plotted with the closest match of MNIST digits according to the mean-squared-error.
		%
		The corresponding fashion-MNIST samples show no continuity nor correlation (despite the intended class correlation).
		%
		e-MNIST shows the desired entanglement for changes of a single latent space factor.
		%
		\textbf{Top-Right}: Latent space of the CVAE for the modalities $\a$ (MNIST) and $\b$ (fashion-MNIST).
		%
		\textbf{Bottom}: Latent space of a trained JMVAE (c.f. Sec. \ref{sec:jmvae_e_mnist_setup}).
		%
		m-MNIST shows clear orthogonalization between modalities of the same class and segregation between classes (colorization is wrt. the CVAE legend).
		%
		e-MNIST shows a coherently learned latent space between the uni- and multi-modal encoders.
		%
		Thus, the JMVAE learns the correlation inside the dataset sufficiently ($\elbo_{\a,\b|\text{me-MNIST}} = -204.48$ vs. $\elbo_{\a,\b|\text{e-MNIST}} = -199.23$).
		}
	\label{fig:e_mnist}
\end{figure} 


%Data continuity We hypothesised that data continuity is important for guiding unsupervised models towards learning the correct data manifolds (Sec. 2). To test this idea we measured how the degree of learnt disentangling changes with reduced continuity in the 2D shapes dataset. We trained a VAE with β = 4 (Fig. 2A) on subsamples of the original 2D shapes dataset, where we progressively decreased the generative factor sampling density. Reduction in data continuity negatively correlates with the average pixel wise (Hamming) distance between two consecutive transforms of each object (normalised by the average number of pixels occupied by each of the two adjacent transforms of an object to account for object scale). Fig. 4A demonstrates that as the continuity in the data reduces, the degree of disentanglement in the learnt representations also drops. This effect holds after additional hyperparameter tuning and can not solely be explained by the decrease in dataset size, since the same VAE can learn disentangled representations from a data subset that preserves data continuity but is approximately 55 \% of the original size (see Sec. 3.4).


%\section{Experiments}
\label{sec:experiment}
%
We apply the datasets explained in Sec. \ref{sec:data_generation} to test and depict the capabilities of the M\textsuperscript{2}VAE.
%
First, we investigate the MoG data comprehensively.
%
Second, we evaluate the ELBO of various approaches to the e-MNIST dataset.
%
The VAEs are compared qualitatively, by visualizing the latent space, and quantitatively by performing lower bound tests $\elbo_{\widetilde{\mathcal{M}}}$ for every subset $\widetilde{\mathcal{M}}\!\subseteq\!\mathcal{M}$ wrt. to the decoding of all modalities $\p_{\theta_{\mathcal{M}}}$:
\begin{align}
%\elbo_{m} = \EX_{q_{\phi_{m}\(z|m\)}}\log\frac{p_{\theta_{m}}\(m|z\)\p\(z\)}{\q_{\phi_{m}}\(z|m\)}\\
%\elbo_{\mathcal{M}} = \EX_{q_{\phi_{\mathcal{M}}\(z|\mathcal{M}\)}}\log\frac{p_{\theta_{\mathcal{M}}}\(\mathcal{M}|z\)\p\(z\)}{\q_{\phi_{\mathcal{M}}}\(z|\mathcal{M}\)}
\elbo_{\widetilde{\mathcal{M}}} = \EX_{q_{\phi_{\widetilde{\mathcal{M}}}\(z|{\widetilde{\mathcal{M}}}\)}}\log\frac{p_{\theta_{\mathcal{M}}}\(\mathcal{M}|z\)\p\(z\)}{\q_{\phi_{{\widetilde{\mathcal{M}}}}}\(z|{\widetilde{\mathcal{M}}}\)}
\label{eq:test_elbo}
\end{align}
with $p\(z\)=\mathcal{N}\(z;\mathbf{0},\mathbf{I}\)$.
%
All VAE architectures can be found in Sec. \ref{sec:jmvae_e_mnist_setup}.
%
%
%\subsection{M\textsuperscript{2}VAE Evaluation}
%
%We perform lower bound tests $\elbo_{\widetilde{\mathcal{M}}}$ for every subset $\widetilde{\mathcal{M}}\!\subseteq\!\mathcal{M}$ wrt. to the decoding of all modalities $\p_{\theta_{\mathcal{M}}}$ to estimate the optimal parameter set $\(\beta_{*},\alpha\)$ of the M\textsuperscript{2}VAE:
%\begin{align}
%\elbo_{m} = \EX_{q_{\phi_{m}\(z|m\)}}\log\frac{p_{\theta_{m}}\(m|z\)\p\(z\)}{\q_{\phi_{m}}\(z|m\)}\\
%\elbo_{\mathcal{M}} = \EX_{q_{\phi_{\mathcal{M}}\(z|\mathcal{M}\)}}\log\frac{p_{\theta_{\mathcal{M}}}\(\mathcal{M}|z\)\p\(z\)}{\q_{\phi_{\mathcal{M}}}\(z|\mathcal{M}\)}
%\elbo_{\widetilde{\mathcal{M}}} = \EX_{q_{\phi_{\widetilde{\mathcal{M}}}\(z|{\widetilde{\mathcal{M}}}\)}}\log\frac{p_{\theta_{\mathcal{M}}}\(\mathcal{M}|z\)\p\(z\)}{\q_{\phi_{{\widetilde{\mathcal{M}}}}}\(z|{\widetilde{\mathcal{M}}}\)}
%\end{align}
%with $p\(z\)=\mathcal{N}\(z;\mathbf{0},\mathbf{I}\)$.
%
\subsection{MoG-Experiment}
\label{sec:experiment_mog}
%
We evaluate the latent space with the premise in mind, that a good generative model should not just generate good data but also gives a good latent representation $z$.
%
\subsubsection{Parametrization}
\begin{figure}
    \def\svgwidth{\textwidth}
	%\tiny
%	\includegraphics[width=\textwidth]{MOoG_MMVAE_setup-0b1000_input.pdf}
	\input{MOoG_MMVAE_setup-0b1000_input_v3.pdf_tex}
	%\includegraphics[width=\textwidth]{MOoG_MMVAE_setup-0b1000_input_v2.pdf}
	%\small
	%\input{MOoG_MMVAE_setup-0b1000_input_v3.pdf_tex}
	%\input{MOoG_MMVAE_setup-0b1000_input.pdf}
	%\input{architecture_RL_application.pdf_tex}
	\caption{Latent space embeddings of the bi-modal MoG dataset by the three encoder networks of the M\textsuperscript{2}VAE.
	%
	Classes and ELBO colorization is depicted for various parameter settings of $\beta_{*}$ and $\alpha$.}
	\label{fig:experiment_mog}
\end{figure}
%
%
We first investigate the impact of the parameter set $\(\beta_{*}, \alpha\)$ on the M\textsuperscript{2}VAE to find a latent space representation, which suits our needs to learn actions from it.

%
As the $\alpha$ parameter controls the mutual connection of all encoders in latent space, we found that a direct connection (i.e. $\alpha = 1.$) puts too much learning pressure on matching the mutual latent distributions between uni- and multi-modal encoders.
%
Thus, classes which should be separated in the multi-modal latent space collapse to the mean distributions of the uni-modal encoders.
%
For $\alpha\!\lessapprox\!10^{-2}$, the encoders are able to find an expressive latent space distribution by means of separable collapsed classes of uni-modal encoders, and expanded classes of multi-modal around it (c.f. Fig. \ref{fig:experiment_mog} top/left).

By the findings of \cite{Higgins2017}, high $\beta$ values result in highly entangled factors in latent space whereas small normalized $\beta_\text{norm}\!\lessapprox\!10^{-2}$ show pretty robust disentanglement in all their test cases.
%
The impact of $\beta$ shows similar behavior on the M\textsuperscript{2}VAE and thus, we chose small $\beta$ values of $\beta_\text{norm}\!=\!10^{-2}$ to relax the learning pressure caused by the prior.
%
While the over optimization wrt. to the prior leads to a fuzzy generation of data $\p\(\mathcal{M} | z \)$ and collapse in latent space, high relaxation ($\beta_\text{norm}\!\ll\!10^{-3}$) causes loss of expressiveness between uni- and multi-modal encoding of a single class by means of the difference in the ELBO.
%

%
%However, just by evaluating the ELBO and choosing the parameter set with the lowest average lower bound, we were able to find a working model with the desired behaviors for the reward signal for the \textbf{action stage (should be renamed)}.
%

It is worth noticing, that diverging $\beta$ parameters between multi- and uni-modal regularization (e.g. $\beta_{\texta\textb} \ll \beta_{\texta}$ or vice versa) results in lower ELBOs, but for the sake of expressiveness of latent embedding and the ELBO between encoders' embeddings.
%
We argue that learning pressure should be applied equally to all encoders so that they experience a similar learning impact.

Another observation results from the fact, that the reconstruction loss of the M\textsuperscript{2}VAE's objective causes learning of mean representatives of classes in the observation space.
This causes the artifact, that if for instance three classes exist in the output space, where one represents the overall mean, and an uni-modal encoder only sees the collapse of classes to that particular mean value, the latent encoding of this uni-modal encoder will collapse to the same mean as well.
%
However, while it is not longer separable (not even non-linearly) in latent space by its mean value, the ELBO for the observation drives up and gives, therefore, evidence about the embedding quality.
%
This insight might be fruitful in terms of epistemic (ambiguity-resolving) tasks, where for instance an unsupervised reinforcement learning approach could use the ELBO as a signal to learn epistemic exploration.
%a later for later Thus, this could if a later action stage wants to learn proper actions from the latent embedding, it needs the embeddings as well es the observation's ELBO.
%
%An important notice is, that the necessary ELBO can be omitted if the input space is factorized, which causes linear separable latent space embeddings between uni- and multi-modal embeddings.
%

\subsubsection{Comparision}
%
%\begin{figure}[h]
%	\def\svgwidth{\textwidth}
%	\tiny
	%\includegraphics[width=\textwidth]{MOoG_MMVAE_setup-0b1000_input.pdf}
	%\includegraphics[width=\textwidth]{MOoG_MMVAE_setup-0b1000_input_v2.pdf}
	%\small
%	\input{MOoG_MMVAE_setup-0b1000_input_v4.pdf_tex}
	%\input{MOoG_MMVAE_setup-0b1000_input.pdf}
	%\input{architecture_RL_application.pdf_tex}
%	\caption{Bi-modal latent space embeddings by the three multi-modal VAEs JMVAE-Zero (top/left), tVAE (bot./left), and M\textsuperscript{2}VAE (top/right). The bi-modal input signal are an arrangement of MoG distributions with ambiguities wrt. their mean values. The ELBO is estimated by Eq. \ref{eq:test_elbo} and is depicted qualitatively, as it can only be compared between encoders of the same approach.}
%	\label{fig:experiment_mog_2}
%\end{figure}
%
\begin{figure}[h]
	\def\svgwidth{\textwidth}
	\tiny
	%\includegraphics[width=\textwidth]{MOoG_MMVAE_setup-0b1000_input.pdf}
	%\includegraphics[width=\textwidth]{MOoG_MMVAE_setup-0b1000_input_v2.pdf}
	%\small
	\input{MOoG_MMVAE_setup-0b1000_input_v5.pdf_tex}
	%\input{MOoG_MMVAE_setup-0b1000_input.pdf}
	%\input{architecture_RL_application.pdf_tex}
	\caption{Bi-modal latent space embeddings by the three multi-modal VAEs JMVAE-Zero (left), tVAE (mid.), and M\textsuperscript{2}VAE (right). The bi-modal input signals are an arrangement of the MoG distributions with ambiguities wrt. their mean values. The ELBO (colorization wrt. Figure \ref{fig:experiment_mog}) is estimated by Eq. \ref{eq:test_elbo} and is depicted qualitatively, as it can only be compared between encoders of the same approach.}
	\label{fig:experiment_mog_2}
\end{figure}
%
%
%
Comparing the three approaches to estimate the multi-modal marginal log-likelihood by maximizing the ELBO, one can see from Fig. \ref{fig:experiment_mog_2} that the most coherent latent space distribution was learned by the proposed M\textsuperscript{2}VAE.

While the JMVAE-Zero learned similarities between $\q_{\phi_{\texta\textb}}$ and $\q_{\phi_{\texta}}$, it learned a complete new embedding for the classes $(1,2)$ with $\q_{\phi_{\textb}}$ (denoted by $(\$)$).
%
Furthermore, the ELBO per embedding allows no conclusion between the embeddings of the various encoders.
%

The tVAE founds a much more coherent embedding between the encoders.
%
This was achieved by the fact, that first the full multi-modal VAE, consisting out of the encoder $\q_{\phi_{\texta\textb}}$ and two decoder $\p_{\theta_{\texta}}$ and $\p_{\theta_{\textb}}$, was trained.
%
Second, the decoder weights are pinned to train the remaining uni-modal networks which enforces coherence.
%
However, the ELBO per embedding also does not allow any direct conclusion between the embeddings of the various encoders. 
%
This is depicted by $(\sim)$, where the multi-modal encoder $\q_{\phi_{\texta\textb}}$ produces embeddings of higher energy than these of the uni-modal ones.
%
This can happen as there is no regularizer which enforces the variational distribution of the encoders to match each other and thus, the KL-divergence may differ between the models for similar encodings.

The M\textsuperscript{2}VAE, on the other hand, enforces the encoders inherently to approximate the same posterior distribution which can be seen by the strong coherence between all embeddings.
%
Furthermore, classes which are separated in the multi-modal latent embedding collapse to the mean values in the uni-modal ones as denoted by $(+)$ and $(-)$.
%
This behavior is also rendered by the ELBO.
%
As the M\textsuperscript{2}VAE makes ambiguous embeddings, the reconstruction loss drives up (c.f. $(*)$ and $(/)$).
%

The embeddings also show an interesting fact about the class $(0)$:
%
As this class is only ambiguously detectable in the uni-modal case, all VAEs learn a linear separable and therefore unambiguous embedding if both modalities make an observation of this class (denoted by $(-)$ for the M\textsuperscript{2}VAE).

%
%For $\alpha\!\lessapprox\!10^{-2}$, the encoders are able find an expressive latent space distributions by means of separable collapsed classes of uni-modal encoders, and expanded classes of multi-modal around it (c.f. Fig. \ref{fig:experiment_mog} top/left).


%the JMVAE-Zero learned similarities between $\q_{\phi_{\texta\textb}}$ and $\q_{\phi_{\texta}}$, it learned a complete new embedding for the classes $(1,2)$ with $\q_{\phi_{\textb}}$, denoted by $(*)$
%
%Furthermore, the ELBO per embedding allows no conclusion between the embeddings of the various encoders.
%


\subsection{In-Place Sensor Fusion}

Further, we introduce the concept of in-place sensor fusion using multi-modal VAEs.
%
This approach is applicable in distributed active-sensing tasks where the latent space representation $z$ of observations $\mathcal{M}'$ (i.e. an object or point of interest was observed by a set of modalities) can be interpreted as inverse sensor model (c.f. \cite{Thrun2005}).
%
This compressed information can be efficiently transmitted between all sensing agents and also be updated as follows:
%
 $z$ can be unfolded to the original observation using the VAE's decoder networks and combined with any new observation $m$ to update the information in-place $z\rightarrow z^{*}$ via
%
\begin{align}
q_{\phi_{m\!\cup\!\mathcal{M}'}}\(z^{*} |m,\mathcal{M}'\)
\quad\text{ with }\quad\mathcal{M}'= \bigcup_{m'in\mathcal{M}'}\!\p_{\theta_{m'}}\( m'|z \)\text{.}
\label{eq:sensor_fusion}
\end{align}
%
% $q_{\phi_{m\!\cup\!\mathcal{M}'}}\(z_n |m\!\cup\!\mathcal{M}'\) = $.
%\autoref{eq:sensor_fusion} performs sensor fusion in-place and is applied as depicted in Fig. \ref{fig:arch}.
%
However, a necessary requirement of Eq. \ref{eq:sensor_fusion} is that auto re-encoding (i.e. $z\rightarrow z$ via $\q_{\phi_{\mathcal{M}'}}\(z |\mathcal{M}'\)$) does not manipulate the information comprised by $z$ in an unrecoverable way (e.g. label-switching).
%
Thus, we assume that VAEs tend to have a natural denoising characteristic (despite the explicit denoising Auto Encoders) which should re-encode any $z$ in a better version of its own by means of the reconstruction loss wrt. $z$.
%
This behavior is shown in Fig. \ref{fig:attractor} where we underlay the latent representation with the reconstruction loss of every particular $z$.
%
One can see the learned discrimination of the latent space by means of high entropy separating the clusters vicinity.
%
Furthermore, initial $z$ values are auto re-encoded which draw the trajectories along their path in latent space.
%
The observable properties of the VAE are that every seed converges to a fixed-point while performing descending steps on the latent space manifold.
%
However, this statement is only valid in general for the proposed M\textsuperscript{2}VAE, as the JMVAE-Zero and tVAE learn no or only similar coherent latent spaces between the encoder networks.
%
Thus, seeds may be attracted by wrong attractors which makes these approach not sufficient for in-place sensor fusion.
%

\begin{figure}[h]
	\def\svgwidth{\textwidth}
	\tiny
	%\includegraphics[width=\textwidth]{MOoG_MMVAE_setup-0b1000_input.pdf}
	%\includegraphics[width=\textwidth]{MOoG_MMVAE_setup-0b1000_input_v2.pdf}
	%\small
	\input{attractor.pdf_tex}
	%\input{MOoG_MMVAE_setup-0b1000_input.pdf}
	%\input{architecture_RL_application.pdf_tex}
	\caption{\textbf{From top to bottom}: JMVAE-Zero, tVAE, and M\textsuperscript{2}VAE. \textbf{Left}: Latent space representation with class colorization. \textbf{Right}: Corresponding colorization of the latent space for every $z$ obtained by auto re-encoding. White dots denote randomly drawn seeds which auto re-encoding steps are represented by the black trajectory. See Fig. \ref{fig:experiment_mog_2} for legends.}
	\label{fig:attractor}
\end{figure}


\subsection{e-MNIST Evaluation}

For this experiment, we estimated the ELBO by Eq. \ref{eq:test_elbo} to evaluate the performance of models JMVAE-Zero, tVAE, and M\textsuperscript{2}VAE.
%
We chose the model wrt. to the evaluation in Fig. \ref{fig:experiment_mog} with $\beta_{\text{norm}}=0.01$ which is $\beta_{*}\approx 4$ for the given MNIST image dimension of $D_{a}=||\(28,28,1\)||$ and $D_{z}=2$.
%
\begin{table}[h]
	\caption{Evidence lower bound test for uni- and multi-modal setups of the VAEs (higher is better).}
	\label{tab:elbo_emnist}
	\centering
	\begin{tabular}{ccc|ccc|ccc}
		\multicolumn{3}{c}{M\textsuperscript{2}VAE} & \multicolumn{3}{c}{tVAE} & \multicolumn{3}{c}{JMVAE-Zero}\\ \hline
		$\elbo_{\a,\b}$ & $\elbo_{\a}$ & $\elbo_{\b}$ & $\elbo_{\a,\b}$ & $\elbo_{\a}$ & $\elbo_{\b}$ & $\elbo_{\a,\b}$ & $\elbo_{\a}$ & $\elbo_{\b}$ \\
		$\mathbf{-10.75}$ & $\mathbf{-10.91}$ & $\mathbf{-16.01}$ & $-23.6$ & $-101.28$ & $-88.75$ & $-24.19$ & $-131.05$ & $-99.71$ 
	\end{tabular}
\end{table}
%
\begin{figure}[h]
	\def\svgwidth{\textwidth}
	\tiny
	%\includegraphics[width=\textwidth]{MOoG_MMVAE_setup-0b1000_input.pdf}
	%\includegraphics[width=\textwidth]{MOoG_MMVAE_setup-0b1000_input_v2.pdf}
	%\small
	\input{emnist_eval.pdf_tex}
	%\input{MOoG_MMVAE_setup-0b1000_input.pdf}
	%\input{architecture_RL_application.pdf_tex}
	\caption{\textbf{From top to bottom}: JMVAE-Zero, tVAE, and M\textsuperscript{2}VAE. \textbf{Left}: Latent space representation with class colorization. \textbf{Right}: Reconstruction from latent space by applying the corresponding decoder networks. $z$ is sampled linearly within $2\sigma$ of the prior for all figures.}
	\label{fig:emnist_eval}
\end{figure}
%
However, Tbl. \ref{tab:elbo_emnist} shows quantitatively and Fig. \ref{fig:emnist_eval} depicts qualitatively that the proposed M\textsuperscript{2}VAE reaches the highest ELBO value, as well as it learns the most expressive latent space distribution.
%
Furthermore, by sampling from the latent space for data generation, the M\textsuperscript{2}VAE reveals crisp reconstructions in comparison to the other approaches.
\begin{confidential}
%
%
\subsection{Robot-Experiment}
\label{sec:experiment_robot}
%
%
\begin{figure}[ht]
	%\def\svgwidth{\textwidth}
	%\includegraphics[width=\textwidth]{picture.png}
	% \small
	%\input{MOoG_MMVAE_setup-0b1000_input.pdf_tex}
	%\input{MOoG_MMVAE_setup-0b1000_input.pdf}
	%\input{architecture_RL_application.pdf_tex}
	\caption{AMiRo robots (left) and Robot-Experiment (right)}
	\label{fig:picture}
\end{figure}
%
\begin{figure*}[ht!]
	\def\svgwidth{0.9\textwidth}
	\footnotesize
	% \small
	%\input{MMVAE_tri_setup-0b1000_combined_bshared-0.01_buni-0.01_a-0.1_g-1.0_e-200.pdf_tex}
	% \input{architecture_RL_application.pdf_tex}
	\caption{\textit{Left}: Visualization of jointly trained latent space embeddings $z$ for all seven encoders $\q_{\phi_{*}}$ of the subsets $\mathcal{P}\(\mathcal{M}\)  \!\setminus\!\emptyset$ with $\mathcal{M}=\left\lbrace\a,\b,\c\right\rbrace$.
		\textit{Left/Top}: Classes that are unambiguously detectable by a subset share similar distribution among all latent spaces (c.f. (1) vs. (a), (a,c), (a,b), and (a,b,c)). 
		%Iff subset $\widetilde{\mathcal{M}}\!\subseteq\!\mathcal{M}$ as learned a good and unambiguous reconstruction (c.f. (c) vs. (4) or (a,b,c) vs. (3)).
		Ambiguous detections from any set $\widetilde{\mathcal{M}}$ collapse to the mean value of the separable classes (c.f. (b,c) and (a,b,c) vs. (3) and (1)).
		\textit{Left/Mid.}: ELBO coloring show high values for ambiguous embeddings and vice versa.
		The ELBO decreases with every modality that joins the subset ((c) vs. (3) $\xrightarrow{\text{add obs. of (a)}}$ (a,c) vs. (3) $\xrightarrow{\text{add obs. of (b)}}$ (a,b,c) vs. (3)), whereas a subset which unambiguously detects a class has already the lowest ELBO value (c.f. (b) vs. (2) $\xrightarrow{\text{add obs. of (a)}}$ (a,b) vs. (2) $\xrightarrow{\text{add obs. of (c)}}$ (a,b,c) vs. (2))
		\textit{Left/Bot.}: The KL-Divergence shows reciprocal behavior to the ELBO, as it has to tighten its shape against the prior to find an specific encoding that allows better reconstruction.
		%
		\textit{Right}: Evolution of the ELBO for jointly trained encoder/decoder networks in a tri-modal setup.
		Minor subsets of $\mathcal{M}$ are always worse because they can never reconstruct ambiguous information (e.g. all encoders $\p_{\theta_{*}}$ are able to reconstruct $a'$, $b'$, and $c'$ if $\q_{\phi_{\texta}}$ encoded (1), but fail on $b'$ and $c'$ if it encodes (2), (3), or (4)).}
	\label{fig:eval_latent_space_tri_modal}
\end{figure*}
%
To train the M\textsuperscript{2}VAE and to obtain the desired behavior of a proper latent distribution, we apply three two-class \textit{support vector machine} (SVM) ($f_\texta$, $f_\textb$, $f_\textc$) to derive feature vectors for each modality with dimensionality $D_\texta\!\equiv\!D_\textb\!\equiv\!D_\textc$ (c.f. \autoref{fig:arch}).
%
Training the M\textsuperscript{2}VAE on raw data resulted in non-sufficient results.
%We conjecture that this effect is caused by the necessarily bigger network architecture which causes the weights in deep layers to collapse \cite{Sonderby2016} and further, by the imbalance in the VAE's reconstruction loss term regarding varying modality dimensionality.
We conjecture that this effect is caused by the necessarily larger network architecture which causes the weights in deep layers to collapse \cite{Sonderby2016} and by the imbalance in the reconstruction loss term for varying modality dimensionality.
%
The experiment was performed on a comprehensive scenario with four classes (c.f. \autoref{tab:class}) and three modalities $\mathcal{M}\!=\!\(\a,\b,\c\)$: 
%
$f_\texta\rightarrow\a$ i.e. a camera based \textit{feature extractor} (FE) that distinguishes between red and green objects;
%
$f_\b\rightarrow\b$ i.e. a LiDAR based FE that distinguishes between round and edgy objects;
%
$f_\c\rightarrow\c$ i.e. a proximity sensor based FE that distinguishes between reflective and mat objects.
%
The M\textsuperscript{2}VAE architecture is stated in Sec. \ref{sec:arcitectures}.
%

The required modality combinations to archive unambiguously classification of an object is shown in \autoref{tab:class}.
%
However, Fig. \ref{fig:eval_latent_space_tri_modal} shows, that the M\textsuperscript{2}VAE is able to detect ambiguous classifications (c.f. Tab. \ref{tab:class}) and is able to develop a coherent relation by means of distribution and ELBO.
%

We define the properties of an MDP to learn epistemic behavior as follows:
$\mathcal{S}_r=\left( z_1, \ldots, z_N, D_{r_1}, \ldots, D_{r_N} \right)$ is the state vector comprising $N$ objects with feature $z$ and the current Euclidean normalized distance $D$ of robot $r$ to each object.
%
We added the distances to encourage myopic behavior, which leads to the fact that first ambiguities of nearer objects are resolved.
%
We allow free communication and thus, the object features $z$ are updated and shared among all robots.
%
$z_n=\(\mu_{1_n},\ldots,\mu_{{D_{z}}_n}, D_{\text{KL}}\)$ comprises the latent embedding of an object plus the current estimated KL-Divergence as a proxy of the ELBO.
%
It is worth mentioning, that calculating the ELBO by Eq. \ref{eq:test_elbo} for real subsets of $\mathcal{M}$ is only possible if the full observation is known.
%
This can only be ensured during training, as the ground truth is known, but not during testing.
%
Therefore, we use the KL-Divergence as a sufficient proxy for the ELBO, as it can directly derived from the sampled statistics ($\mu$ and $\sigma$) of the encoder network.
%

Each object $n$ can be observed by taking one action $a_n$, or the episode can be terminated before observing all objects by the selection of \textit{no operation} (NOP): $\mathcal{A}=\left( a_1, \ldots, a_N, \text{NOP} \right)$, while dependent on the robots modality $m$, an action $a_n$ samples from the posterior $q_{\phi_{m}}\(z_n | m_n\)$.
%
However, if there is a former observation of the object $n$ by any other set of modalities $\mathcal{M}'$, the former latent embedding $z'_n$ is first encoded after which the robot applies a multi-modal encoder
%
\begin{align}
  q_{\phi_{m\!\cup\!\mathcal{M}'}}\(z_n |m,\mathcal{M}'\)
  \text{ with }\mathcal{M}'=\bigcup_{m'in\mathcal{M}'}\!\p_{\theta_{m'}}\( m'|z'_n \)
  \label{eq:sensor_fusion}
\end{align}
%
% $q_{\phi_{m\!\cup\!\mathcal{M}'}}\(z_n |m\!\cup\!\mathcal{M}'\) = $.
\autoref{eq:sensor_fusion} performs sensor fusion in-place and is applied as depicted in Fig. \ref{fig:arch}.

% each other's modality, the encoder $q_{\phi}(z_n | x_n, w_n )$ is applied to perform sensor-fusion in place by generating the missing modality using $p_{\theta_{x}}(z_n)$ or $p_{\theta_{w}}(z_n)$ (c.f. \autoref{fig:sensor_fusion}).

%$\mathcal{M}'\)$, with $\mathcal{M}'$ being the set of modalities which have seen the object $n$ before.

We train a modality dependent deep Q-Network (c.f. architecture and parameters in Sec. \ref{sec:arcitectures}) to estimate the action $a_r$ which maximizes the ELBO decrease, given the current state $\mathcal{S}_r$.
%
Therefore, the reward $\mathcal{R}$ is defined by the increase of negative ELBO after observation of object $n$ by $\mathcal{R} \propto \Delta \mathcal{L}_n=\mathcal{L}'_n - \mathcal{L}_n$.
%
Thus, the reward is shaped as follows: $\mathcal{R}=\Delta \mathcal{L}_n + (1 - D_n)$ if the observation led to a decrease of the ELBO; $\mathcal{R}=-1$ if there is no decrease; quitting the exploration by NOP results in $\mathcal{R}=0$.
%
%For each robot, or more precisely each modality, an own Q-Network is trained.

It is worth mentioning that there is no interaction between the agents and thus, the learning process is decoupled since all robots act independently of one another.
%
Furthermore, we estimate the distances during training geometrically and during testing in Gazebo in real life via ROS \texttt{move\_base}.
%
This decouples the training from performing real actions and allows the efficient sampling of object and robot positions to generate data which increases training time drastically.
%
As a matter of fact, the feature extractors $f_{*}$ abstracts the sensing in simulation and reality which allows zero-shot domain adaptation \cite{Higgins2017}.
%
%\footnote{Path planning and low level control is done by ROS \texttt{move\_base}}.
%
%
%
%
%We performed experiments using three AMiRo mini-robots equipped with a camera, LiDAR, and proximity sensors \cite{Herbrechtsmeier2016a} to classify objects in the environment.
%
%The overall mapping and decision architecture is depicted in \autoref{fig:architecture_network}.
%
%Every robot explores the environment for objects and encodes the detection to a common map that is shared among all robots.
%
%Based on this environmental model, a modality specific DQN decides which robot has to pursue which object to increase the information in the map.
%
%It is worth mentioning, that the DQN only selects the target objects and has no spatial information nor direct control at the motor level\footnote{Path planning and low level control is done by ROS \texttt{move\_base}}.
%We use the DQN algorithm by \cite{Mnih2015} for training the deep Q-Network and sample the test and training data from the Gazebo simulator.

%The number of possible states of $\mathcal{M}\in\mathcal{S}$ is $\left|\mathcal{S}\right|=\#\text{PoI}!\cdot\#\text{PoI-encoding}^{\#\text{PoI} \#\text{mod.}}$, which is 384 for this relatively comprehensible experiment assuming only binary PoI-encodings.
%Having continues encodings, as offered by the JMVAE, the task of controlling the robots by the means of handcrafted architectures based on the encodings becomes unfeasible.
%Therefore, we train a DQN for which we select $z_n=\left(\left(\mu_{1,n},\sigma_{1,n}\right), \ldots,\left(\mu_{D_{z},n}, \sigma_{D_{z},n}\right)\right)\in \mathcal{M}$ and $I_n=1/\left\lVert \left( \sigma_{1,n}, \ldots, \sigma_{{D_{z}},n}\right)\right\rVert_2$.
%
%The reward is shaped as follows: $r=\Delta I_n$ if the observation led to a decrease of the ELBO; $r=-1$ if there is no decrease; quitting the exploration by NOP results in $r=0$.
%The Network architecture and training parameters are listed in Sec. \ref{}
%
We evaluated the training by the total reward the agent collects in an episode averaged over a 512 randomly sampled environments.
The average total reward metric is shown in Fig. \ref{fig:reward}; it demonstrates the successful adaptation of each modality's network to our task (c.f. application video\footnote{\url{https://goo.gl/ZuMwmb}}).
%
However, Fig. \ref{fig:reward} reveals no information about the epistemic behavior, as the reward comes from the unsupervised trained VAEs.
%
Thus, we measured the accuracy of all robots, taking the next most informative action (i.e. independent of distance) which we were able to evaluate due to the comprehensiveness of the experiment.
%
\autoref{tab:class} reveals that training a DQN by the proposed M\textsuperscript{2}VAE leads to the highest accuracy.
%
Eliminating the leverage of the distance dependent reward leads to an almost perfect behavior which remains only affected by the noisy feature extractors.
%
%The derived parameter set from the MoG-Experiment work pretty robust even for the tri-modal case, 
%
\begin{table}
	\caption{Left: List of required modalities to perform unambiguous classification. Right: Accuracy of taking optimal actions to resolve ambiguity w/ and w/o a distance depended reward $\mathcal{R}$ over $1000$ experiments.}
	\footnotesize
	\begin{tabular}{cc}
	\begin{tabular}{r c c c}
		\textbf{class vs. modality}&$\a$&$\b$&$\c$\\
		\hline			
		green, mat, cyl. (1) & \checkmark &            & \\
		red, mat, cube   (2) &            & \checkmark & \\
		red, mat, cyl.   (3) & \checkmark & \checkmark & \checkmark\\
		red, shiny, cyl. (4) &            &            & \checkmark
	\end{tabular}
	&
	\begin{tabular}{r c c}
		\textbf{Accuracy} & w/ & w/o \\
		\hline			
		JMVAE-Z.  & $65.99$,  & $69.78$ \\
		tVAE      & $71.16$   & $81.38$ \\
		M\textsuperscript{2}VAE     & $\mathbf{82.89}$   & $\mathbf{95.55}$ \\
		          &           & 
	\end{tabular}
    \end{tabular}
	\label{tab:class}
\end{table}
%
%\begin{table}
%	\caption{\juxi{taking a swing at this}
%		For each of the four objects in our experiments different modalities are required to identify. Object 1, for example, is the only green object and is therefore uniquely identifiable with the colour sensor. Object 3 on the other hand requires observation by all three modalities to identify. \juxi{a picture of the four object might help too}}
%	\footnotesize
%	\begin{tabular}{ r | c l | c l | c l }
%		\textbf{class vs.~modality} & \multicolumn{2}{c}{Colour} & \multicolumn{2}{c}{Geometry} & \multicolumn{2}{c}{Texture} \\
%		\hline			
%		Object 1      & green & \checkmark & cylinder      &        & matte &  \\ % green, mat, cylinder (
%		Object 2      & red   &            & cube      & \checkmark & matte & \\ % red, mat, cube       (
%		Object 3      & red   & \checkmark & cylinder  & \checkmark & matte & \checkmark \\% red, mat, cube       (
%		Object 4      & red   &            & cylinder   &           & shiny & \checkmark %          & \checkmark % red, shiny, cylinder (
%	\end{tabular}
%	\label{tab:class}
%\end{table}
%
%
%
%\textbf{Other than in the statement by Higgins \cite{Higgins2017}, data for classification needs to be non-continues.
%For instance, shapes like cylinder, ovals, and edges which can be transiently represented by one factor in latent space, are not hard categorize.
%Discontinues features, which also generate multiple factors in latent space, are better for detection.}
%
\begin{figure}[ht]
	%\def\svgwidth{\textwidth}
	%\includegraphics[width=0.6\textwidth]{reward_during_training.png}
	% \small
	%\input{MOoG_MMVAE_setup-0b1000_input.pdf_tex}
	%\input{MOoG_MMVAE_setup-0b1000_input.pdf}
	%\input{architecture_RL_application.pdf_tex}
	\caption{Average reward during training.}
	\label{fig:reward}
\end{figure}
%
\end{confidential}

%\section{Conclusion}
\label{sec:conclusion}
%
This work presents a novel multi-modal Variational Auto Encoder which is derived from the complete marginal joint log-likelihood.
%
We showed that this expression can jointly be trained on an Mixture-of-Gaussian dataset with ambiguous observations, as well as on a complex dataset derived from MNIST and fashion-MNIST.
%
Furthermore, we formulated requirements and characteristics for multi-modal data for sensor fusion and derived a technique to learn new datasets, namely the proposed entangled-MNIST, which suffice these requirements.
%
Lastly, we developed the idea of in-place sensor fusion in distributed, active sensing scenarios and formulated the requirements, by means of auto re-encoding, to VAEs.
%
This revealed the properties of VAEs, that they tend to denoise the observable data which leads to an attractor behavior in latent space.
%
However, we performed all qualitative evaluations of the latent space with the premise in mind, that a good generative model should not just generate good data but also gives a good latent representation.
%
This does also correlate with the quantitative behaviors, as our proposed model achieved the highest ELBO values.
%
Future work will concentrate on the integration of the ambiguous resolving characteristics to an epistemic-exploration scenario.

\subsubsection*{Acknowledgments}

This research was supported by 'CITEC' (EXC 277) at Bielefeld University and the Federal Ministry of Education and Research (57388272). The responsibility for the content of this publication lies with the author.

%\bibliography{iclr2019_conference}
\bibliography{root}
\bibliographystyle{iclr2019_conference}

\section*{APPENDIX}

\subsection{Extension to three Modalities}
\label{seq:extension_three_mod_suzuki}
The proposed, as well as approach by \cite{Suzuki2017}, can be extended to multiple modalities $\mathcal{M}\!=\!\left\lbrace\a,\b,c\right\rbrace$.
The conditional marginal log-likelihood of $a$ can be written as
\begin{align}
\log\p\( \a | \b,\c\)\!=\!\elboTMa \!+\! \dkl \( \q \( z | \mathcal{M} \) \| \p\( z| \mathcal{M} \) \)\!\geq\!\elboTMa\text{.}
\label{eq:elbo3_conditional_geq}
\end{align}
%
\subsubsection{JMVAE for three Modalities}
%
The VI between a set of distributions $\mathcal{M}$ can be written as $-\EX_{\p\(\mathcal{M}\)}\sum_{m\in\mathcal{M}} \log\p\(m|\mathcal{M}\setminus m\)$, which leads to an expression of maximizing the ELBO of negative VI (c.f. \cite{Suzuki2017}).
Following this approach, the log-likelihood $L_{\text{3M}}$ can be expressed by the ELBOs, by utilizing Eq. \ref{eq:elbo3_conditional_geq}, of their conditionals and KL divergence:
\begin{align}
L_\text{3M} &= \log\p\( \a | \b,\c\) + \log\(\p\( \b | \a,\c\)\) + \log\(\p\( \c | \b,\c\)\) \\
%&= \elboTMa + \elboTMb + \elboTMc + 3 \dkl \( \q \( z | \a,\b,\c \) \| \p\( z| \a,\b,\c \) \) \\
&\geq \elboTMa + \elboTMb + \elboTMc \\
&\geq \elboTJ -\dkl \( \q \( z | \a,\b,\c \) \| \p\( z| \b,\c \) \) \\
& - \dkl \( \q \( z | \a,\b,\c \) \| \p\( z| \a,\c \) \) 
\!-\!\dkl \( \q \( z | \a,\b,\c \) \| \p\( z| \b,\c \) \) %\label{eq:multi_vae_loss}
\label{eq:elbo3_joint_geq}
\end{align}
%The combined ELBO $\elbo_{\text{3M}}$ can then be rewritten as
%\begin{align}
%&\elbo_{\text{3M}} = \\
%& \EX_{\q\( z | \a,\b,\c \)} \log \( \p \( \a | z \)  \) - \dkl \( \q \( z | \a,\b,\c \) \| \p\( z| \b,\c \) \) \\
%& +
%\EX_{\q\( z | \a,\b,\c \)} \log \( \p \( \b | z \)  \) - \dkl \( \q \( z | \a,\b,\c \) \| \p\( z| \a,\c \) \) \\
%& +
%\EX_{\q\( z | \a,\b,\c \)} \log \( \p \( \c | z \)  \) - \dkl \( \q \( z | \a,\b,\c \) \| \p\( z| \a,\b \) \)\\
%&\geq \elboTJ -\dkl \( \q \( z | \a,\b,\c \) \| \p\( z| \b,\c \) \) \\
%& - \dkl \( \q \( z | \a,\b,\c \) \| \p\( z| \a,\c \) \) \\
%& - \dkl \( \q \( z | \a,\b,\c \) \| \p\( z| \b,\c \) \) \label{eq:multi_vae_loss}
%\end{align}
with $\elboTJ$ being the joint ELBO of a joint probability $\p\(\mathcal{M}\)$ which expression is analog to Eq. \ref{eq:joint_reg_rec_rec}.
%
\subsubsection{M\textsuperscript{2}VAE for three Modalities}
%
Applying the proposed scheme to the joint log-likelihood of three modalities results in the following expression:
%
\begin{align}
&L_{\text{3M\textsuperscript{2}}} \\
%
&=\! \nicefrac{3}{3} \log\p\( \a,\b,\c \) = \nicefrac{1}{3} \log\p\( \a,\b,\c \)^3 & \\
%
&=\! \nicefrac{1}{3} \log\p\( \a,\b,\c \)\p\( \a,\b,\c \)\p\( \a,\b,\c \) \\
%
&=\! \nicefrac{1}{3} \log\p\(\a,\b \)\p\(\b,\c \)\p\(\a,\c \)\p\( \a|\b,\c \)\p\( \b|\a,\c \)\p\( \c|\a,\b \) \\
%
&=\! \nicefrac{1}{3} \(\log\(\p\(\a,\b \) \)+\log\( \p\(\b,\c \) \)+\log\( \p\(\a,\c \) \) \rightEmptyBrace \\
&\phantom{=}\! \leftEmptyBrace +\log \p\( \a|\b,\c \) +\log \p\( \b|\a,\c \)+\log \p\( \c|\a,\b \)\) \\
%
&=\! \nicefrac{1}{3} \(\nicefrac{2}{2}\(\log\p\(\a,\b \) +\log \p\(\b,\c \)+\log\p\(\a,\c \) \)\!+\!L_{\text{3M}}\) \\
&=\! \nicefrac{1}{6}\(\log\p\(\a,\b \)^2 +\log \p\(\b,\c \)^2 +\log \p\(\a,\c \)^2 \)\! +\! \nicefrac{L_{\text{3M}}}{3} \\
&=\! \nicefrac{1}{6}\(L_{\text{M\textsuperscript{2}}_{\texta\textb}} +L_{\text{M\textsuperscript{2}}_{\textb\textc}} +L_{\text{M\textsuperscript{2}}_{\texta\textc}} \) + \nicefrac{1}{3}L_{\text{3M}}
\end{align}
%
From here on, one can substitute all log-likelihoods given the expressions in Sec. \ref{sec:models} and \ref{sec:appendix}, to derive the ELBO $\elbo_{\text{3M\textsuperscript{2}}}$.
%

\subsubsection{M\textsuperscript{2}VAE Derivation}
\label{sec:proof_mmvae}

%For the sake of brevity, we shorten the marginal joint log-likelihood of the set $\mathcal{M}$ of variables as $L_{\mathcal{M}}:=\log \p\(\mathcal{M}\)$
%
%\begin{align}
%&L_{\text{M\textsuperscript{2}}_{\mathcal{M}}} = \nicefrac{|\mathcal{M}|}{|\mathcal{M}|} L_{\mathcal{M}}
%%
%=\! \nicefrac{1}{|\mathcal{M}|} \sum_{m\in\mathcal{M}} L_{m | \mathcal{M} \setminus m } + L_{\mathcal{M} \setminus m } =\! \nicefrac{1}{|\mathcal{M}|} \( L_{\text{M}_{\mathcal{M}}} + \sum_{\widetilde{m}\in\widetilde{\mathcal{M}}} L_{\text{M\textsuperscript{2}}_{\widetilde{m}}} \)
%\end{align}
%
\begin{align}
L_{\text{M\textsuperscript{2}}_{\mathcal{M}}} &= \log \p\(\mathcal{M} \) \overset{\text{mul. 1}}{=}
%
\nicefrac{|\mathcal{M}|}{|\mathcal{M}|} \log \p\(\mathcal{M} \)
%
\overset{\text{log. mul.}}{=} 
\nicefrac{1}{|\mathcal{M}|} \log \p\(\mathcal{M} \)^{|\mathcal{M}|}
%
\\
%
&\overset{\text{Bayes}}{=} 
\nicefrac{1}{|\mathcal{M}|} \sum_{m\in\mathcal{M}} \log \p\(\mathcal{M} \setminus m  \) \p\(m | \mathcal{M} \setminus m  \)
%
\\
&\overset{\text{log. add}}{=} 
\nicefrac{1}{|\mathcal{M}|} \sum_{m\in\mathcal{M}} \log \p\(\mathcal{M} \setminus m  \) + \log \p\(m | \mathcal{M} \setminus m  \)
\end{align}
%
The expression $\sum_{m\in\mathcal{M}} \log \p\(m | \mathcal{M} \setminus m  \)$ is the general form of the marginal log-likelihood for the \textit{variation of information} (VI), as introduced by \cite{Suzuki2017} for the JMVAE, for any set $\mathcal{M}$.
%
Thus, it can be directly substituted with $L_{\text{M}_{\mathcal{M}}}$.
%
The expression $\sum_{m\in\mathcal{M}} \log \p\(\mathcal{M} \setminus m  \)$ is the combination of all joint log-likelihoods of the subsets of $\mathcal{M}$ which have one less element.
%
Therefore, this term can be rewritten as 
\begin{align}
\sum_{m\in\mathcal{M}} \log \p\(\mathcal{M} \setminus m  \) = \sum_{\widetilde{m}\in\widetilde{\mathcal{M}}} \log \p\( \widetilde{m} \)
\end{align}
with $\widetilde{\mathcal{M}}=\left\lbrace m | m \in \mathcal{P}\( \mathcal{M}\), |m|=|\mathcal{M}|-1 \right\rbrace$
%
Finally, $\log \p\( \widetilde{m} \)$ can be substituted by $L_{\text{M\textsuperscript{2}}_{\widetilde{m}}} $ without loss of generality.
%
However, it is worth noticing that substitution stops at the end of recursion and therefore, all final expressions $\log \p\( \widetilde{m} \)\ \forall \ |\widetilde{m}|\equiv1$ remain. $ \square $
%\begin{align}
%%
%%
%= \nicefrac{1}{|\mathcal{M}|} \sum_{|\mathcal{M}|}\log \p\(\mathcal{M} \)
%%
%= \nicefrac{1}{|\mathcal{M}|} \sum_{|\mathcal{M}|}\log \p\(\mathcal{M} \)
%%
%\\
%&= \nicefrac{1}{|\mathcal{M}|} \sum_{m\in\mathcal{M}} L_{m | \mathcal{M} \setminus m } + L_{\mathcal{M} \setminus m } =\! \nicefrac{1}{|\mathcal{M}|} \( L_{\text{M}_{\mathcal{M}}} + \sum_{\widetilde{m}\in\widetilde{\mathcal{M}}} L_{\text{M\textsuperscript{2}}_{\widetilde{m}}} \)
%\end{align}
\subsubsection{Network Architecture}
\label{sec:jmvae_e_mnist_setup}

We designed all VAEs such that the latent space prior is given by a Gaussian with unit variance.
%
Furthermore, all VAEs sample from a Gaussian variational distribution that is parametrized by the encoder networks.
%
A summary of all architectures used in this paper can be seen in Tbl. \ref{tab:architectures}.
%
The reconstruction loss for calculating the evidence lower bound was performed by \textit{binary cross-entropy} (BCE) for the e-MNIST and \textit{root-mean-squared error} (RMS) for the MoG experiment.
%
\begin{table}[h]
	\caption{Various VAE architectures and optimizers for the e-MNIST and MoG experiments. um/mm stand for uni- and multi-modal while fc refers to fully-connected layers.}
	\label{tab:architectures}
	\centering
	\begin{tabular}{llllc}
		\textbf{Issue} & \textbf{VAE} & \textbf{Optimizer} &  & \textbf{VAE architecture} \\ \hline
        \multirow{2}{*}{e-MNIST}        & \multirow{2}{*}{JMVAE-Z.} & \multirow{2}{*}{adam} & encoder & fc 2x784-2x128-2x64-concat-64-2 (ReLU) \\
		 &   &  & decoder & fc 2x64-2x128-2x786 (tanh) \\ \hline
		        &            &      & um enc. & fc 784-128-64-2 (ReLU) \\
		e-MNIST & tVAE       & adam & mm enc. & fc 2x784-2x128-2x64-concat-64-2 (ReLU) \\
		        &            &      & decoder & fc 2x64-2x128-2x786 (tanh) \\ \hline
		        &            &      & um enc. & fc 784-128-64-2 (ReLU) \\
		e-MNIST & M\textsuperscript{2}VAE & adam & mm enc. & fc 2x784-2x128-2x64-concat-64-2 (ReLU) \\
		        &            &      & decoder & fc 2x64-2x128-2x786 (tanh) \\ \hline
		\multirow{2}{*}{MoG} & \multirow{2}{*}{JMVAE-Z.} & \multirow{2}{*}{rmsprop} & encoder & fc 2x2-2x128-concat-64-2 (ReLU) \\
		    &            &         & decoder & fc 2x128-2x2 (tanh) \\ \hline
		    &            &         & um enc. & fc 2x2-2x128-2x2 (ReLU) \\
		MoG & tVAE       & rmsprop & mm enc. & fc 2x2-2x128-concat-64-2 (ReLU) \\
		    &            &         & decoder & fc 2x128-2x2 (tanh) \\ \hline
	  	    &            &         & um enc. & fc 2-128-2 (ReLU) \\
		MoG & M\textsuperscript{2}VAE & rmsprop & mm enc. & fc 2x2-2x128-concat-64-2 (ReLU) \\
		    &            &         & decoder & fc 2x128-2x2 (tanh) \\ 
	\end{tabular}
\end{table}

Furthermore, the CVAE for training the e-MNIST dataset is designed as depicted in Tbl. \ref{tab:architectures_emnist}.

\begin{table}[h]
	\caption{CVAE architecture for each dataset MNIST and fashion-MNIST. The label as one-hot-vector is concatenated after the convolution layers and fed into the fully-connected (fc) layers. For convolutional architectures the numbers in parenthesis indicate strides, while padding is always \textit{same}.}
	\label{tab:architectures_emnist}
	\centering
	\begin{tabular}{lc}
		        & \textbf{CVAE architecture} \\ \hline
		encoder & conv 1x2x2-64x2x2 (2)-64x3x3-64x3x3-concat label C-fc 128-2\\
		decoder & concat label C-fc 128-deconv reverse of encoder (ReLU)
	\end{tabular}
\end{table}



\end{document}
